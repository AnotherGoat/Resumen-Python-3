\documentclass{report}

\usepackage[utf8]{inputenc}
\usepackage[a4paper, left=3cm, right=3cm, top=2.5cm, bottom=2.5cm]{geometry}
\usepackage[outputdir=build]{minted}
\usepackage{sectsty}
\usepackage{tocloft}
\usepackage{graphicx}
\usepackage[dvipsnames]{xcolor}
\usepackage{amsmath}
\usepackage[spanish]{babel}
\usepackage[colorlinks=true, linkcolor=., urlcolor=NavyBlue]{hyperref}

% estilo de los segmentos de código
\usemintedstyle{perldoc}

% capítulos de color azul y secciones de color rojo
\chapterfont{\color{OliveGreen}}
\sectionfont{\color{RoyalBlue}}

% también en la tabla de contenidos
\renewcommand{\cftchapfont}{\bfseries\color{OliveGreen}}
\renewcommand{\cftsecfont}{\color{RoyalBlue}}

% párrafos sin sangría
\setlength\parindent{0pt}
% separación entre párrafos
\setlength\parskip{1.5ex}

% cambiar espacio entre números y texto en la tabla de contenidos
%\setlength\cftchapternumwidth{4em}
%\setlength\cftsectionnumwidth{4em}

% comandos nuevos para comillas simples y dobles
\newcommand{\simple}[1]{`#1'}
\newcommand{\doble}[1]{``#1''}

% color nuevo para los bloques de código
\definecolor{grisclaro}{RGB}{240, 240, 240}

% macros nuevas para minted
\newmintedfile{python}{bgcolor=grisclaro, frame=single, linenos, numbersep=6pt}
\newmintedfile{text}{frame=single}

\title{Resumen Python 3}
\author{Víctor Mardones Bravo}
\date{Febrero de 2021}

\begin{document}

% la primera hoja no tiene números de página
\pagenumbering{gobble}

% brujería para centrar el título y logo
\null
\nointerlineskip
\vfill
\let\snewpage \newpage
\let\newpage \relax
% se centra el logo svg exportado con Inkscape
  {\centering\def\svgwidth{\columnwidth}
  \input{logo\\python-logo-inkscape.pdf_tex}}
\maketitle
\let \newpage \snewpage
\vfill 
\break

\clearpage

% tabla de contenido
\tableofcontents

\clearpage

% el resto del documento tiene números de página
\pagenumbering{arabic}

\chapter{Introducción a Python}

\section{¿Qué es Python?}

\href{https://www.python.org}{\underline{Python}} es un lenguaje de programación de alto nivel, con aplicaciones en numerosas áreas, incluyendo programación web, scripting, computación científica e inteligencia artificial.

Es muy popular y usado por organizaciones como \href{https://www.google.com}{\underline{Google}}, \href{https://www.nasa.gov}{\underline{la NASA}}, \href{https://www.cia.gov}{\underline{la CIA}} y \href{https://www.disney.com}{\underline{Disney}}.

No hay limitaciones en lo que se puede construir usando Python. Esto incluye aplicaciones autónomas, aplicaciones web, juegos, ciencia de datos, modelos de machine learning y mucho más.

Dato curioso: Según el creador Guido van Rossum, el nombre de Python viene de la serie de comedia británica \doble{El Circo Volador de Monty Python}.

\section{El Zen de Python}

El Zen de Python es una colección de 19 \doble{principios} para escribir programas de computadores que influenciaron el diseño y representan la filosofía de este lenguaje de programación.

El Zen de Python se muestra en pantalla la primera vez que se ejecute la siguiente linea.

\pythonfile{codigo/1-intro/import_this.py}

Después se mostrará el siguiente texto.

\textfile{codigo/1-intro/zen.txt}

\section{Hola mundo}

Para mostrar el texto \doble{Hola mundo} en pantalla se puede usar la función print().

\pythonfile{codigo/1-intro/hola_mundo_1.py}

Cada declaración de impresión print() genera texto en una nueva línea.

\pythonfile{codigo/1-intro/hola_mundo_2.py}

\clearpage\chapter{Conceptos básicos}

\section{Comentarios}

Los comentarios son anotaciones en el código utilizadas para hacerlo más fácil de entender. No afectan la ejecución del código.

En Python, los comentarios comienzan con el símbolo \#. Todo el texto luego de este \# (dentro de la misma línea) es ignorado.

\pythonfile{codigo/2-basico/comentario.py}

Python no soporta comentarios multilínea, al contrario de otros lenguajes de programación.

A lo largo de este resumen, se usarán comentarios para mostrar la salida de algunas funciones, cuando sea posible.

\section{Operaciones aritméticas}

Python tiene la capacidad de realizar cálculos. Los operadores +, -, * y / representan suma, resta, multiplicación y división, respectivamente.

\pythonfile{codigo/2-basico/aritmetica_1.py}

Los espacios entre los signos y los números son opcionales, pero hacen que el código sea más fácil de leer.

\section{La regla PEMDAS}

Las operaciones en Python siguen el orden dado por la regla PEMDAS:

\begin{enumerate}
  \item Paréntesis
  \item Exponentes
  \item Multiplicación y división
  \item Adición y Sustracción
\end{enumerate}

% TODO: aritmetica 2 con ejemplos de operaciones combinadas

\section{Paréntesis}

Se pueden usar paréntesis () para agrupar operaciones y hacer que estas se realicen primero, siguiendo la regla PEMDAS.

\pythonfile{codigo/2-basico/aritmetica_3.py}

\section{Floats}
    
Para representar números racionales o que no son enteros, se usa el tipo de dato float o punto flotante. Se pueden crear directamente ingresando un número con un punto decimal, o como resultado de una división.

\pythonfile{codigo/2-basico/float_1.py}

Se debe tener en cuenta que los computadores no pueden almacenar perfectamente el valor de los floats, lo cual a menudo conduce a errores.

\pythonfile{codigo/2-basico/error_float.py}

El error mostrado arriba es un error clásico de la aritmética de punto flotante. Incluso tiene su propio \href{\underline{https://0.30000000000000004.com}}{sitio web}.

Al trabajar con floats, no es necesario escribir un 0 a la izquierda del punto decimal.

\pythonfile{codigo/2-basico/float_2.py}

Esta notación se asemeja a decir \doble{punto cinco} en vez de \doble{cero punto cinco}.

El resultado de cualquier operación entre floats o entre un float y un entero siempre dará como resultado un float. La división entre enteros también da como resultado un float.

%TODO: float 3 mostrando resultados de operaciones entro floats y enteros

\section{Exponenciación}

Otra operación soportada es la exponenciación, que es la elevación de un número a la potencia de otro. Esto se realiza usando el operador **.

Ejemplo equivalente a $2 ^ 5 = 32$.

\pythonfile{codigo/2-basico/potencia.py}

\section{Cociente y resto}

La división entera se realiza usando el operador //, donde el resultado es la parte entera que queda al realizar la división, también conocida como cociente.

La división entera retorna un entero en vez de un float.

\pythonfile{codigo/2-basico/cociente.py}

Para obtener el resto al realizar una división entera, se debe usar el operador módulo \%.

\pythonfile{codigo/2-basico/resto.py}

Esta operación es equivalente a $7 \mod{2}$ en aritmética modular.

Este operador viene de la aritmética modular, y uno de sus usos más comunes es para saber si un número es múltiplo de otro. Esto se hace revisando si el módulo al dividirlo por ese otro número es 0.

\pythonfile{codigo/2-basico/uso_del_resto.py}

El caso particular módulo 2 también puede usarse para saber si un número es par o no.

\clearpage\chapter{Cadenas de texto}

\section{Strings o cadenas de caracteres}

Las cadenas de caracteres se crean introduciendo el texto entre comillas simples $'$$'$ o dobles $''$$''$.

\pythonfile{codigo/3-strings/string.py}

Un string debe empezar y terminar con comillas del mismo tipo, no se permiten comillas mixtas.

\pythonfile{codigo/3-strings/strings_no_validos.py}

\section{Cadena vacía}

A veces es necesario inicializa un string, pero sin agregarle información. Una cadena vacía es definida como $'$$'$ o $''$$''$.

\pythonfile{codigo/3-strings/cadena_vacia.py}

Estos strings vacíos se inicializan en variables, lo cual se verá en el capítulo siguiente.

\section{Caracteres especiales}

Algunos caracteres no se pueden incluir directamente en una cadena. Para esos casos, se debe incluir la barra diagonal inversa \textbackslash antes de ellos.

\pythonfile{codigo/3-strings/caracteres_especiales.py}

Los caracteres $'$, $''$ y \textbackslash son especiales, porque normalmente cumplen funciones especiales dentro de strings.

Si el string se define entre comillas dobles, no es necesario poner $'$ para ingresar comillas simples dentro de él, y viceversa.

\pythonfile{codigo/3-strings/comillas.py}

\section{Secuencias de escape}

Las secuencias de escape también se pueden incluir usando el símbolo \textbackslash dentro de cadenas de texto. Su origen viene de las secuencias de escape usadas en las máquinas de escribir.

Algunas de las secuencias de escape más usadas son:
% TODO: mejorar estos ejemplos

\begin{itemize}
  \item Nueva línea (new line): Avanza una línea hacia adelante (salto de línea) y deja el cursor al principio de esta línea (retorno de carro). Representado por \textbackslash n.
  
  \pythonfile{codigo/3-strings/newline.py}
  
  Cualquier caracter después de \textbackslash n queda en la línea siguiente.

  \item Tabulador horizontal (horizontal tab): Añade un salto de tabulador horizontal. Representado por \textbackslash t.
  
  \pythonfile{codigo/3-strings/tab.py}
  
  El salto de tabulador avanza hasta el siguiente \doble{tab stop} de la misma línea.

  \item Retorno de carro (carriage return): Mueve el \doble{carro} (cursor) al principio de la línea actual, eliminando todos los caracteres de esa línea. Representado por \textbackslash r.
  
  \pythonfile{codigo/3-strings/carriage_return.py}

  \item Retroceso (backspace): Borra el último carácter y mueve el cursor al carácter anterior. Representado por \textbackslash b.
  
  \pythonfile{codigo/3-strings/backspace.py}

\end{itemize}

Otras secuencias de escape que cada vez se usan menos son:

\begin{itemize}
  \item Tabulador vertical (vertical tab): Añade un salto de tabulador vertical. Representado por \textbackslash v.
  
  \pythonfile{codigo/3-strings/vtab.py}

  La tabulación vertical avanza hasta la siguiente línea que sea una \doble{tab stop}.

  \item Salto de página (form feed): Baja a la próxima \doble{página}. Representado por \textbackslash f.
  
  \pythonfile{codigo/3-strings/form_feed.py}
  
  Algunos programadores los usaban para separar distintas secciones de código en \doble{páginas}.

\end{itemize}

\section{Caracteres Unicode}

Las barras diagonales inversas también se pueden usar para escribir caracteres Unicode arbitrarios. Se escriben como \textbackslash u seguido del código del carácter Unicode (en hexadecimal).

Los códigos Unicode se aceptan sin importar que tengan mayúsculas o minúsculas.

\pythonfile{codigo/3-strings/unicode.py}

El \href{https://unicode.org/}{\underline{sitio web de Unicode}} contiene más información sobre estos caracteres y sobre este estándar. \href{https://unicode-table.com/en/}{\underline{Este sitio web}} tiene una tabla con los códigos.

\section{Strings multilínea}

Este es un tipo especial de string, que se escribe entre comillas triples $'$$'$$'$ o $''$$''$$''$, y que reconoce los saltos de línea sin necesidad de usar la secuencia \textbackslash n.

\pythonfile{codigo/3-strings/string_multilinea.py}

\section{Comentarios multilínea}

Los comentarios multilínea no existen formalmente en Python, pero se puede hacer algo parecido usando

% TODO: Esta sección

\section{Concatenación de strings}

\section{Multiplicación de strings}

\section{Opciones del método print()}

\clearpage\chapter{Variables}

\section{Asignación de variables}

\section{Nombre de variables válidos}

\section{Palabras clave}

\section{Operaciones con variables}

\section{Entrada}

\section{Conversión de tipos de datos}

\section{Operadores de asignación}

\clearpage\chapter{Declaraciones if}

\section{Booleanos}

\section{Operadores de comparación}

\section{Declaración if}

\section{Declaración if-else}

\section{Declaración elif}

\section{Operadores lógicos}

\section{Precedencia de operadores lógicos}

\clearpage\chapter{Listas}

\section{Creación de listas}

\section{Indexación de listas}

\section{Lista vacía}

\section{Anidación de listas}

\section{Operaciones con listas}

\section{Funciones de listas}

\section{Copiar listas}

\section{Strings como listas}

\section{Indexación de strings}

\section{Conversión de strings a listas}

\clearpage\chapter{Bucles}

\section{Bucles while}

\section{Declaración break}

\section{Declaración continue}

\section{Bucle for con listas}

\section{Rangos}

\section{Bucle for en rangos}

\clearpage\chapter{Funciones}

\section{¿Qué es una función?}

\section{Definición de funciones}

\section{Llamado de funciones}

\section{Devolución de valores en una función}

\section{Docstring}

\section{Funciones como objetos}

\section{Sobrecarga de funciones}

\section{Anotaciones de tipos}

\clearpage\chapter{Módulos y la biblioteca estándar}

\section{Módulos}

\section{Alias}

\section{La biblioteca estándar}

\section{Módulos externos y pip}

\clearpage\chapter{El módulo math}

\clearpage\chapter{El módulo random}

\clearpage\chapter{Manejo de excepciones}

\section{Excepciones}

\section{Declaración try-except}

\section{Declaración finally}

\section{Levatar excepciones}

\section{Aserciones}

\clearpage\chapter{Pruebas unitarias}

\clearpage\chapter{Manejo de archivos}

\section{Abrir archivos}

\section{Modos de apertura}

\section{Cierre de archivos}

\section{Lectura de archivos}

\section{Escritura de archivos}

\section{Declaración with}

\clearpage\chapter{Módulos time y datetime}

\clearpage\chapter{Iterables}

\section{Objeto None}

\section{Diccionarios}

\section{Indexación de diccionarios}

\section{Uso de in y not en diccionarios}

\section{Función get()}

\section{Función keys()}

\section{Tuplas}

\section{Cortes de lista}

\section{Cortes de tuplas}

\section{Subcadenas}

\section{Listas por compresión}

\section{Formateo de cadenas}

\section{Funciones de cadenas}

\section{Funciones all() y any()}

\section{Función enumerate()}

\clearpage\chapter{Programación funcional}

\section{Funciones puras}

\section{Lambdas}

\section{Función map()}

\section{Función filter()}

\section{Generadores}

\section{Decoradores}

\section{Iteración}

\section{Recursión}

\clearpage\chapter{Conjuntos y estructuras de datos}

\section{Conjuntos}

\section{Operaciones con conjuntos}

\section{Estructuras de datos}

\clearpage\chapter{El módulo itertools}

\section{Iteradores infinitos}

\section{Operaciones sobre iterables}

\section{Funciones de combinatoria}

\clearpage\chapter{Programación orientada a objetos}

\section{Programación orientada a objetos}

\section{Clases}

\section{Método init}

\section{Atributos}

\section{Métodos}

\section{Atributos de clase}

\section{Excepciones de clases}

\section{Herencia}

\section{Función super()}

\section{Métodos mágicos}

\section{Sobrecarga de operadores aritméticos}

\section{Sobrecarga de operadores de comparación}

\section{Métodos mágicos de contenedores}

\section{Ciclo de vida de un objeto}

\section{Encapsulamiento}

\clearpage\chapter{Expresiones regulares}

\clearpage\chapter{Empaquetamiento}

\clearpage\chapter{Interfaz gráfica}

\clearpage\chapter{Algoritmos de ordenamiento}

\clearpage\chapter{Algoritmos de búsqueda}

\clearpage\chapter{Algoritmos de matrices}

\clearpage\chapter{Implementación de estructuras de datos}

\clearpage\chapter{La librería NumPy}

\end{document}