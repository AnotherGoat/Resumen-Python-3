\documentclass{report}

\usepackage[utf8]{inputenc}
\usepackage[a4paper, left=3cm, right=3cm, top=2.5cm, bottom=2.5cm]{geometry}
\usepackage[outputdir=build]{minted}
\usepackage{sectsty}
\usepackage{tocloft}
\usepackage{graphicx}
\usepackage[dvipsnames]{xcolor}
\usepackage{amsmath}
\usepackage[spanish]{babel}
\usepackage[colorlinks=true, linkcolor=., urlcolor=NavyBlue]{hyperref}

% estilo de los segmentos de código
\usemintedstyle{perldoc}

% capítulos de color azul y secciones de color rojo
\chapterfont{\color{OliveGreen}}
\sectionfont{\color{RoyalBlue}}

% también en la tabla de contenidos
\renewcommand{\cftchapfont}{\bfseries\color{OliveGreen}}
\renewcommand{\cftsecfont}{\color{RoyalBlue}}

% párrafos sin sangría
\setlength\parindent{0pt}
% separación entre párrafos
\setlength\parskip{1.5ex}

% cambiar espacio entre números y texto en la tabla de contenidos
%\setlength\cftchapternumwidth{4em}
%\setlength\cftsectionnumwidth{4em}

% comandos nuevos para comillas simples y dobles
\newcommand{\simple}[1]{`#1'}
\newcommand{\doble}[1]{``#1''}

% color nuevo para los bloques de código
\definecolor{grisclaro}{RGB}{240, 240, 240}

% macros nuevas para minted
\newmintedfile{python}{bgcolor=grisclaro, frame=single, linenos, numbersep=6pt}
\newmintedfile{text}{frame=single}

\title{Resumen Python 3}
\author{Víctor Mardones Bravo}
\date{Febrero de 2021}

\begin{document}

% la primera hoja no tiene números de página
\pagenumbering{gobble}

% brujería para centrar el título y logo
\null
\nointerlineskip
\vfill
\let\snewpage \newpage
\let\newpage \relax
% se centra el logo svg exportado con Inkscape
  {\centering\def\svgwidth{\columnwidth}
  \input{logo\\python-logo-inkscape.pdf_tex}}
\maketitle
\let \newpage \snewpage
\vfill 
\break

\clearpage

% tabla de contenido
\tableofcontents

\clearpage

% el resto del documento tiene números de página
\pagenumbering{arabic}

\chapter{Introducción a Python}

\section{¿Qué es Python?}

\href{https://www.python.org}{\underline{Python}} es un lenguaje de programación de alto nivel, con aplicaciones en numerosas áreas, incluyendo programación web, scripting, computación científica e inteligencia artificial.

Es muy popular y usado por organizaciones como \href{https://www.google.com}{\underline{Google}}, \href{https://www.nasa.gov}{\underline{la NASA}}, \href{https://www.cia.gov}{\underline{la CIA}} y \href{https://www.disney.com}{\underline{Disney}}.

No hay limitaciones en lo que se puede construir usando Python. Esto incluye aplicaciones autónomas, aplicaciones web, juegos, ciencia de datos, modelos de machine learning y mucho más.

Dato curioso: Según el creador Guido van Rossum, el nombre de Python viene de la serie de comedia británica \doble{El Circo Volador de Monty Python}.

\section{El Zen de Python}

El Zen de Python es una colección de 19 \doble{principios} para escribir programas de computadores que influenciaron el diseño y representan la filosofía de este lenguaje de programación.

El Zen de Python se muestra en pantalla la primera vez que se ejecute la siguiente linea.

\pythonfile{codigo/1-intro/import_this.py}

Después se mostrará el siguiente texto.

\textfile{codigo/1-intro/zen.txt}

\section{Hola mundo}

Para mostrar el texto \doble{Hola mundo} en pantalla se puede usar la función print().

\pythonfile{codigo/1-intro/hola_mundo_1.py}

Cada declaración de impresión print() genera texto en una nueva línea.

\pythonfile{codigo/1-intro/hola_mundo_2.py}

\clearpage\chapter{Conceptos básicos}

\section{Comentarios}

Los comentarios son anotaciones en el código utilizadas para hacerlo más fácil de entender. No afectan la ejecución del código.

En Python, los comentarios comienzan con el símbolo \#. Todo el texto luego de este \# (dentro de la misma línea) es ignorado.

\pythonfile{codigo/2-basico/comentario.py}

Python no soporta comentarios multilínea, al contrario de otros lenguajes de programación.

A lo largo de este resumen, se usarán comentarios para mostrar la salida de algunas funciones, cuando sea posible.

\section{Operaciones aritméticas}

Python tiene la capacidad de realizar cálculos. Los operadores +, -, * y / representan suma, resta, multiplicación y división, respectivamente.

\pythonfile{codigo/2-basico/aritmetica_1.py}

Los espacios entre los signos y los números son opcionales, pero hacen que el código sea más fácil de leer.

\section{La regla PEMDAS}

Las operaciones en Python siguen el orden dado por la regla PEMDAS:

\begin{enumerate}
  \item Paréntesis
  \item Exponentes
  \item Multiplicación y división
  \item Adición y Sustracción
\end{enumerate}

% TODO: aritmetica 2 con ejemplos de operaciones combinadas

\section{Paréntesis}

Se pueden usar paréntesis () para agrupar operaciones y hacer que estas se realicen primero, siguiendo la regla PEMDAS.

\pythonfile{codigo/2-basico/aritmetica_3.py}

\section{Floats}
    
Para representar números racionales o que no son enteros, se usa el tipo de dato float o punto flotante. Se pueden crear directamente ingresando un número con un punto decimal, o como resultado de una división.

\pythonfile{codigo/2-basico/float_1.py}

Se debe tener en cuenta que los computadores no pueden almacenar perfectamente el valor de los floats, lo cual a menudo conduce a errores.

\pythonfile{codigo/2-basico/error_float.py}

El error mostrado arriba es un error clásico de la aritmética de punto flotante. Incluso tiene su propio \href{\underline{https://0.30000000000000004.com}}{sitio web}.

Al trabajar con floats, no es necesario escribir un 0 a la izquierda del punto decimal.

\pythonfile{codigo/2-basico/float_2.py}

Esta notación se asemeja a decir \doble{punto cinco} en vez de \doble{cero punto cinco}.

El resultado de cualquier operación entre floats o entre un float y un entero siempre dará como resultado un float. La división entre enteros también da como resultado un float.

%TODO: float 3 mostrando resultados de operaciones entro floats y enteros

\section{Exponenciación}

Otra operación soportada es la exponenciación, que es la elevación de un número a la potencia de otro. Esto se realiza usando el operador **.

Ejemplo equivalente a $2 ^ 5 = 32$.

\pythonfile{codigo/2-basico/potencia.py}

\section{Cociente y resto}

La división entera se realiza usando el operador //, donde el resultado es la parte entera que queda al realizar la división, también conocida como cociente.

La división entera retorna un entero en vez de un float.

\pythonfile{codigo/2-basico/cociente.py}

Para obtener el resto al realizar una división entera, se debe usar el operador módulo \%.

\pythonfile{codigo/2-basico/resto.py}

Esta operación es equivalente a $7 \mod{2}$ en aritmética modular.

Este operador viene de la aritmética modular, y uno de sus usos más comunes es para saber si un número es múltiplo de otro. Esto se hace revisando si el módulo al dividirlo por ese otro número es 0.

\pythonfile{codigo/2-basico/uso_del_resto.py}

El caso particular módulo 2 también puede usarse para saber si un número es par o no.

\clearpage\chapter{Cadenas de texto}

\section{Strings o cadenas de caracteres}

Las cadenas de caracteres se crean introduciendo el texto entre comillas simples $'$$'$ o dobles $''$$''$.

\pythonfile{codigo/3-strings/string.py}

Un string debe empezar y terminar con comillas del mismo tipo, no se permiten comillas mixtas.

\pythonfile{codigo/3-strings/strings_no_validos.py}

\section{Cadena vacía}

A veces es necesario inicializa un string, pero sin agregarle información. Una cadena vacía es definida como $'$$'$ o $''$$''$.

\pythonfile{codigo/3-strings/cadena_vacia.py}

Estos strings vacíos se inicializan en variables, lo cual se verá en el capítulo siguiente.

\section{Caracteres especiales}

Algunos caracteres no se pueden incluir directamente en una cadena. Para esos casos, se debe incluir la barra diagonal inversa \textbackslash antes de ellos.

\pythonfile{codigo/3-strings/caracteres_especiales.py}

Los caracteres $'$, $''$ y \textbackslash son especiales, porque normalmente cumplen funciones especiales dentro de strings.

Si el string se define entre comillas dobles, no es necesario poner $'$ para ingresar comillas simples dentro de él, y viceversa.

\pythonfile{codigo/3-strings/comillas.py}

\section{Secuencias de escape}

Las secuencias de escape también se pueden incluir usando el símbolo \textbackslash dentro de cadenas de texto. Su origen viene de las secuencias de escape usadas en las máquinas de escribir.

Algunas de las secuencias de escape más usadas son:
% TODO: mejorar estos ejemplos

\begin{itemize}
  \item Nueva línea (new line): Avanza una línea hacia adelante (salto de línea) y deja el cursor al principio de esta línea (retorno de carro). Representado por \textbackslash n.
  
  \pythonfile{codigo/3-strings/newline.py}
  
  Cualquier caracter después de \textbackslash n queda en la línea siguiente.

  \item Tabulador horizontal (horizontal tab): Añade un salto de tabulador horizontal. Representado por \textbackslash t.
  
  \pythonfile{codigo/3-strings/tab.py}
  
  El salto de tabulador avanza hasta el siguiente \doble{tab stop} de la misma línea.

  \item Retorno de carro (carriage return): Mueve el \doble{carro} (cursor) al principio de la línea actual, eliminando todos los caracteres de esa línea. Representado por \textbackslash r.
  
  \pythonfile{codigo/3-strings/carriage_return.py}

  \item Retroceso (backspace): Borra el último carácter y mueve el cursor al carácter anterior. Representado por \textbackslash b.
  
  \pythonfile{codigo/3-strings/backspace.py}

\end{itemize}

Otras secuencias de escape que cada vez se usan menos son:

\begin{itemize}
  \item Tabulador vertical (vertical tab): Añade un salto de tabulador vertical. Representado por \textbackslash v.
  
  \pythonfile{codigo/3-strings/vtab.py}

  La tabulación vertical avanza hasta la siguiente línea que sea una \doble{tab stop}.

  \item Salto de página (form feed): Baja a la próxima \doble{página}. Representado por \textbackslash f.
  
  \pythonfile{codigo/3-strings/form_feed.py}
  
  Algunos programadores los usaban para separar distintas secciones de código en \doble{páginas}.

\end{itemize}

\section{Caracteres Unicode}

Las barras diagonales inversas también se pueden usar para escribir caracteres Unicode arbitrarios. Se escriben como \textbackslash u seguido del código del carácter Unicode (en hexadecimal).

Los códigos Unicode se aceptan sin importar que tengan mayúsculas o minúsculas.

\pythonfile{codigo/3-strings/unicode.py}

El \href{https://unicode.org/}{\underline{sitio web de Unicode}} contiene más información sobre estos caracteres y sobre este estándar. \href{https://unicode-table.com/en/}{\underline{Este sitio web}} tiene una tabla con los códigos.

\section{Strings multilínea}

Este es un tipo especial de string, que se escribe entre comillas triples $'$$'$$'$ o $''$$''$$''$, y que reconoce los saltos de línea sin necesidad de usar la secuencia \textbackslash n.

\pythonfile{codigo/3-strings/string_multilinea.py}

\section{Comentarios multilínea}

Los comentarios multilínea no existen formalmente en Python, pero se puede hacer algo parecido usando

% TODO: Esta sección

\section{Concatenación de strings}

Dos o más cadenas se pueden unir una después de la otra, usando un proceso llamado concatenación. Se usa el operador +.

%TODO

La concatenación sólo se puede realizar entre strings, no entre cadenas y números.

%TODO

\section{Multiplicación de strings}

Las cadenas también pueden ser multiplicadas por números enteros. Esto produce una versión repetida de la cadena original. El orden de la cadena y el número no importa, pero la cadena suele ir primero.

%TODO

También se pueden combinar operaciones de multiplicación y concatenación.

%TODO

Multiplicar por 0 genera un string vacío.

%TODO

\section{Opciones del método print()}

El método print() puede aceptar más de un string como argumento, lo cual hace que se muestren en una misma línea separados por espacios.

%TODO

El método print() tiene 2 argumentos que pueden definirse para dar más control sobre lo que se imprime en la pantalla.

El argumento sep define el string separador entre cada string \doble{normal} que se le entregue al método print(), excepto después del último. Estos separadores pueden incluir secuencias de escape.

%TODO

El argumento end define el string que irá después del último string \doble{normal}. Se puede usar para seguir escribiendo en la misma línea, si no se incluye la secuencia de escape \textbackslash n.

%TODO
%TODO

Ambos argumentos se pueden combinar.

%TODO

\clearpage\chapter{Variables}

\section{Asignación de variables}

Una variable permite almacenar un valor asignándole un nombre, el cual puede ser usado para referirse al valor más adelante en el programa. Para asignar una variable, se usa el signo igual =.

%TODO

\section{Nombre de variables válidos}

Se aplican ciertas restricciones con respecto a los caracteres que se pueden usar en los nombres de variables. Los únicos caracteres permitidos son letras, números y guiones bajos. Además, no se puede comenzar con números o incluir espacios.

%TODO

Python es sensible a mayúsculas y minúsculas, lo que significa que las variables \doble{num}, \doble{Num}, \doble{NUM}, etc. son distintas.

\section{Palabras clave}

Existen palabras específicas que tampoco se pueden usar como nombres de variables. El intérprete de Python reconoce estas palabras como palabras clave o keywords, y tienen usos reservados.

A continuación, se muestra una lista de todas las palabras clave en Python.

%TODO

Nótese el uso de mayúsculas al principio de False, None y True.

\section{Operaciones con variables}

Se pueden usar variables dentro de operaciones. Lo único que se debe recordar es que deben declararse antes.

%TODO

Una variable también puede cambiar de valor a lo largo de la ejecución de un programa.

%TODO

\section{Entrada}

Para obtener información del usuario, se puede usar la función input(). La información obtenida puede ser almacenada como una variable.

%TODO

Toda la información recibida por el método input() es retornada como un string.
También se puede entregar un string como parámetro al método input(), lo cual mostrará texto antes de pedir la entrada. Esto sirve para aclarar qué entrada está solicitando el programa.

%TODO

Al usar la función input(), el flujo del programa se detiene hasta que el usuario ingrese algún valor.

\section{Conversión de tipos de datos}

\section{Operadores de asignación}

Los operadores de asignación permiten escribir código como \simple{x = x + 1} de manera más concisa, como \simple{x += 1}. Lo mismo es posible con otros operadores como -, *, /, //, \% y **.

%TODO

También se pueden usar con los operadores de concatenación y multiplicación de strings.

%TODO

\clearpage\chapter{Declaraciones if}

\section{Booleanos}

\section{Operadores de comparación}

\section{Declaración if}

\section{Declaración if-else}

\section{Declaración elif}

\section{Operadores lógicos}

\section{Precedencia de operadores lógicos}

\clearpage\chapter{Listas}

\section{Creación de listas}

\section{Indexación de listas}

\section{Lista vacía}

\section{Anidación de listas}

\section{Operaciones con listas}

\section{Funciones de listas}

\section{Copiar listas}

\section{Strings como listas}

\section{Indexación de strings}

\section{Conversión de strings a listas}

\clearpage\chapter{Bucles}

\section{Bucles while}

\section{Declaración break}

\section{Declaración continue}

\section{Bucle for con listas}

\section{Rangos}

\section{Bucle for en rangos}

\clearpage\chapter{Funciones}

\section{¿Qué es una función?}

Cualquier sentencia que consista de una palabra seguida de información entre paréntesis es llamada una función.

Ejemplos de funciones que se han visto anteriormente.

%TODO

\section{Definición de funciones}

\section{Llamado de funciones}

\section{Devolución de valores en una función}

\section{Docstring}

\section{Funciones como objetos}

\section{Sobrecarga de funciones}

\section{Anotaciones de tipos}

\clearpage\chapter{Módulos y la biblioteca estándar}

\section{Módulos}

\section{Alias}

\section{La biblioteca estándar}

Hay 3 tipos principales de módulos en Python: aquellos que escribes tú mismo, aquellos que se instalan de fuentes externas y aquellos que vienen preinstalados con Python.

El último tipo se denomina la biblioteca estándar, y contiene muchos módulos útiles. Algunos de estos módulos son:

%TODO

La extensa biblioteca estándar de Python es una de sus principales fortalezas como lenguaje. Se puede encontrar más información sobre los módulos de la biblioteca estándar en \href{https://docs.python.org/3/library/index.html}{\underline{la documentación}}.

Algunos de los módulos en la biblioteca estándar están escritos en Python y otros en C. La mayoría están disponibles en todas las plataformas, pero algunos son específicos de Windows o Unix.

\section{Módulos externos y pip}

Muchos módulos de Python creados por terceros son almacenados en el índice de paquetes Python (Python Package Index, PyPI). Se puede ver el repositorio en su \href{https://pypi.org}{\underline{sitio web oficial.}}

La mejor manera de instalar estos es utilizando un programa llamado pip. Este viene instalado por defecto con las distribuciones modernas de Python.

Para instalar una biblioteca, se debe buscar su nombre, ir a la línea de comandos y escribir pip install nombre.

%TODO

Es importante recordar que los comandos de pip se deben introducir en la línea de comandos, no en el interpretador de Python.

Se puede ingresar el comando pip help para ver información sobre otros comandos que se pueden usar con este gestor de paquetes.

Utilizar pip es la forma estándar de instalar bibliotecas en la mayoría de sistemas operativos, pero algunas bibliotecas tienen binarios predefinidos para Windows. Estos son archivos ejecutables regulares que permiten instalar bibliotecas con una interfaz gráfica de la misma manera que se instalan otros programas.

\clearpage\chapter{El módulo math}

\clearpage\chapter{El módulo random}

\clearpage\chapter{Manejo de excepciones}

\section{Excepciones}

\section{Declaración try-except}

\section{Declaración finally}

\section{Levatar excepciones}

\section{Aserciones}

\clearpage\chapter{Pruebas unitarias}

\clearpage\chapter{Manejo de archivos}

\section{Abrir archivos}

\section{Modos de apertura}

\section{Cierre de archivos}

\section{Lectura de archivos}

\section{Escritura de archivos}

\section{Declaración with}

\clearpage\chapter{Módulos time y datetime}

\clearpage\chapter{Iterables}

\section{Objeto None}

%TODO: Mover a un capítulo más apropiado

\section{Diccionarios}

%TODO: Método keys()

\section{Indexación de diccionarios}

\section{Uso de in y not en diccionarios}

\section{Función get()}

\section{Función keys()}

\section{Tuplas}

\section{Cortes de lista}

\section{Cortes de tuplas}

\section{Subcadenas}

\section{Listas por compresión}

\section{Formateo de cadenas}

\section{Funciones de cadenas}

\section{Funciones all() y any()}

\section{Función enumerate()}

\clearpage\chapter{Programación funcional}

\section{Funciones puras}

\section{Lambdas}

\section{Función map()}

\section{Función filter()}

\section{Generadores}

\section{Decoradores}

\section{Iteración}

\section{Recursión}

\clearpage\chapter{Conjuntos y estructuras de datos}

\section{Conjuntos}

\section{Operaciones con conjuntos}

\section{Estructuras de datos}

\clearpage\chapter{El módulo itertools}

\section{Iteradores infinitos}

\section{Operaciones sobre iterables}

\section{Funciones de combinatoria}

\clearpage\chapter{Programación orientada a objetos}

%TODO: mejorar la redacción en todo el capítulo

\section{Programación orientada a objetos}

Anteriormente se vieron 2 paradigmas de programación: imperativa (utilizando declaraciones, bucles y funciones como subrutinas) y funcional (utilizando funciones puras, funciones de orden superior y recursión).

Otro paradigma muy popular es la programación orientada a objetos (POO). Los objetos son creados utilizando clases, las cuales son en realidad el eje central de la POO.

\section{Clases}

La clase describe lo que el objeto será, pero es independiente del objeto mismo. En otras palabras, una clase puede ser descrita como los planos, la descripción o definición de un objeto. Una misma clase puede ser utilizada como plano para crear varios objetos diferentes.

Las clases son creadas utilizando la palabra clave class y un bloque indentado que contiene los métodos de una clase (los cuales son funciones).

%TODO

El código define una clase llamada Gato, la cual tiene el atributo color. Luego, la clase es utilizada para crear 2 objetos independientes de esa clase.

\section{Método \_\_init\_\_}

El método \_\_init\_\_ es el más importante de una clase. Es llamada cuando una instancia (objeto) de una clase es creada, utilizando el nombre de la clase como función.

Todos los métodos deben tener self como su primer parámetro. Aunque no sea pasado explícitamente, Python agrega el argumento self automáticamente. No se necesita entregar cuando se llaman los métodos.

Dentro de la definición de un método, self se refiere a la instancia que está llamando al método.

Las instancias de una clase tienen atributos, los cuales son datos asociados a ellas. En este ejemplo, las instancias de Gato tienen los atributos color y edad. Los atributos pueden ser accedidos al poner un punto seguido del nombre del atributo luego del nombre de una instancia.

%TODO

En un método \_\_init\_\_, self.atributo puede ser usado para fijar un valor inicial a los atributos de una instancia.

En el método mostrado anteriormente, el método \_\_init\_\_ recibe 2 argumentos y los asigna a los atributos del objeto. El método \_\_init\_\_ es llamado el constructor de la clase.

Si una clase no tiene atributos que se quieran inicializar con cada instancia, se puede omitir el método \_\_init\_\_.

%TODO

%TODO Separar esta sección

\section{Atributos}

\section{Métodos}

Las clases pueden tener otros métodos definidos para agregarles funcionalidad. Todos los métodos deben tener self como su primer parámetro.
Estos métodos son accedidos utilizando la misma sintaxis de punto que los atributos.

%TODO

\section{Atributos de clase}

Las clases pueden tener atributos de clase también, creados al asignar variables dentro del cuerpo de una clase. Estos pueden ser accedidos desde instancias de una clase o desde la clase misma.

%TODO

Los atributos de clase son compartidos por todas las instancias de una clase. Realizar algún cambio a un atributo de la clase también hará ese cambio en las instancias de esa clase.

%TODO

\section{Excepciones de clases}

Tratar de acceder a un atributo de una instancia que no está definida generará un AttributeError. Esto también aplica cuando se llama un método no definido.

%TODO

\section{Herencia}

La herencia brinda una manera de compartir funcionalidades entre clases.

Por ejemplo, las clases Perro, Gato, Conejo, etc. tienen algo en común. Aunque presenten algunas diferencias, también tienen muchas características en común. Este parecido puede ser expresado haciendo que todos hereden de una superclase Animal, que contiene las funcionalidades compartidas.

Para heredar de una clase desde otra, se coloca el nombre de la superclase entre paréntesis luego del nombre de la clase.

%TODO

Una clase que hereda de otra clase se llama subclase. Una clase de la cual se hereda se llama superclase.

Si una clase hereda de otra con los mismos atributos o métodos, los sobreescribe.

%TODO

La herencia también puede ser indirecta. Una clase hereda de otra, y esa clase puede a su vez heredar de una tercera clase.

%TODO

Sin embargo, no es posible la herencia circular.

\section{Función super()}

La función super() es una útil función relacionada con la herencia que hace referencia a la clase padre. Puede ser utilizada para encontrar un método con un determinado nombre en la superclase del objeto.

%TODO

También se puede usar para llamar al constructor de la superclase.

%TODO

\section{Métodos mágicos}

Los métodos mágicos son métodos especiales que tienen doble guión bajo al principio y al final de sus nombres. Son también conocidos en inglés como dunders (de double underscores).

El constructor \_\_init\_\_ es un método mágico, pero existen muchos más. Son utilizados para crear funcionalidades que no pueden ser representadas en un método regular.

\section{Sobrecarga de operadores aritméticos}

\section{Sobrecarga de operadores de comparación}

Python también ofrece métodos mágicos para comparaciones.

%TODO

Si \_\_ne\_\_ no está implementado, devuelve el opuesto de \_\_eq\_\_. No hay ninguna otra relación entre los otros operadores.

%TODO

\section{Métodos mágicos de contenedores}

Hay varios métodos mágicos para hacer que las clases actúen como contenedores.

%TODO

A continuación, se muestra un ejemplo rebuscado pero creativo, el cual consiste en crear una clase de lista poco confiable o imprecisa.

%TODO

%TODO: Investigar más métodos mágicos

\section{Ciclo de vida de un objeto}

El ciclo de vida de un objeto está conformado por su creación, manipulación y destrucción.

La primera etapa del ciclo de vida de un objeto es la definición de la clase a la cual pertenece.

La siguiente etapa es la instanciación de un objeto, cuando el método \_\_init\_\_ es llamado. La memoria es asignada para almacenar la instancia. Justo antes de que esto ocurra, el método \_\_new\_\_ de la clase es llamado, para asignar la memoria necesaria. Este es normalmente redefinido sólo en casos especiales.

Luego de que ocurra lo anterior el objeto estará listo para ser utilizado.

Otro código puede interactuar con el objeto, llamando sus métodos o accediendo a sus atributos. Eventualmente, terminará de ser utilizado y podrá ser destruido.

Cuando un objeto es destruido, la memoria asignada se libera y puede ser utilizada para otros propósitos.

La destrucción de un objeto ocurre cuando su contador de referencias llega a cero. La cuenta de referencias es el número de variables y otros elementos que se refieren al objeto.

Si nada se está refiriendo al objeto (tiene una cuenta de referencias de 0) nada puede interactuar con este, así que puede ser eliminado con seguridad.
En algunas situaciones, dos (o más) objetos pueden solo referirse entre ellos, y por lo tanto pueden ser eliminados también.

La sentencia del reduce la cuenta de referencias de un objeto por 1, y a menudo conlleva a su eliminación. El método mágico de la sentencia del es \_\_del\_\_.

El proceso de eliminación de objetos cuando ya no son necesarios se denomina recolección de basura (garbage collection).

En resumen, el contador de referencias de un objeto se incrementa cuando se le es asignado un nuevo nombre o es colocado en un contenedor (una lista, tupla o diccionario). La cuenta de referencias de un objeto se disminuye cuando es eliminado con del, su referencia es reasignada, o su referencia sale fuera del alcance. Cuando la cuenta de referencias de un objeto llega a 0, Python lo elimina automáticamente.

%TODO

Lenguajes de bajo nivel como C no tienen esta clase de manejo de memoria automático.

\section{Encapsulamiento}

Un componente clave de la programación orientada a objetos es el encapsulamiento, que involucra empaquetar las variables y funciones relacionadas en un único objeto fácil de usar, una instancia de una clase.

Un concepto asociado es el de ocultamiento de información, el cual dicta que los detalles de implementación de una clase deben estar ocultos y que sean presentados a aquellos que quieran utilizar la clase en una interfaz estándar limpia. En otros lenguajes de programación, esto se logra normalmente utilizando métodos y atributos privados, los cuales bloquean el acceso externo a ciertos métodos y atributos en una clase.

La filosofía de Python es ligeramente diferente. A menudo se dice “todos somos adultos consistentes aquí”, que significa que no deberías poner restricciones arbitrarias al acceso de las partes de una clase. Por ende, no hay formas de imponer que un método o atributo sea estrictamente privado.

Sin embargo, hay maneras de desalentar a la gente de acceder a las partes de una clase, tales como denotar que es un detalle de implementación y debe ser utilizado a su cuenta y riesgo.


\clearpage\chapter{Expresiones regulares}

\clearpage\chapter{Empaquetamiento}

\clearpage\chapter{Interfaz gráfica}

\clearpage\chapter{Algoritmos de ordenamiento}

\clearpage\chapter{Algoritmos de búsqueda}

\clearpage\chapter{Algoritmos de matrices}

\clearpage\chapter{Implementación de estructuras de datos}

\clearpage\chapter{La librería NumPy}

\end{document}