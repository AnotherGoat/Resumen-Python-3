\chapter{Estructuras de datos}

\section{Estructuras de datos}

%TODO: esta sección

\section{Diccionarios}

Los diccionarios son estructuras de datos utilizadas para mapear claves arbitrarias a valores. Pueden ser indexados de la misma manera que las listas, utilizando corchetes que contengan claves.

\pythonfile{codigo/16-estructuras/diccionarios_1.py}
\out{codigo/16-estructuras/diccionarios_1.out}

Cada elemento de un diccionario es representado por un par clave:valor (key:value). Los elementos dentro de diccionarios no están ordenados, lo que significa que no pueden ser indexados por índices, sólo por claves.\smallskip

Un diccionario puede almacenar como valor cualquier tipo de datos. Se pueden usar saltos de línea para definirlos de forma más ordenada.

\pythonfile{codigo/16-estructuras/diccionarios_2.py}
\out{codigo/16-estructuras/diccionarios_2.out}

\section{Listas como diccionarios}

Las listas pueden ser consideradas como diccionarios con claves de números enteros dentro de un cierto rango.

\pythonfile{codigo/16-estructuras/listas_como_diccionarios.py}
\out{codigo/16-estructuras/listas_como_diccionarios.out}

\section{Diccionario vacío}

Un diccionario vacío es definido como \ttt{\{\}}.

\pythonfile{codigo/16-estructuras/diccionario_vacio.py}

\section{Excepciones en diccionarios}

Tratar de indexar una clave que no es parte de un diccionario retorna un \ttt{KeyError}.

\pythonfile{codigo/16-estructuras/error_diccionario.py}
\out{codigo/16-estructuras/error_diccionario.out}

Sólo objetos inmutables pueden ser utilizados como claves de diccionario. Los objetos inmutables son aquellos que no pueden ser cambiados. Algunos objetos mutables son listas, conjuntos y diccionarios. Tratar de utilizar un objeto mutable como clave de diccionario ocasiona un \ttt{TypeError}.

\pythonfile{codigo/16-estructuras/error_clave_mutable.py}
\out{codigo/16-estructuras/error_clave_mutable.out}

\section{Indexación de diccionarios}

Al igual que las listas, las claves de un diccionario pueden ser asignadas a distintos valores. Sin embargo, a diferencia de las listas, se le puede asignar un valor a nuevas claves, no sólo a las que ya existen.

\pythonfile{codigo/16-estructuras/indexacion_diccionarios.py}
\out{codigo/16-estructuras/indexacion_diccionarios.out}

\section{Uso de in y not en diccionarios}

Para determinar si una clave está en un diccionario, se puede usar los operadores in y not in, al igual que en listas.

\pythonfile{codigo/16-estructuras/in_not_diccionarios.py}
\out{codigo/16-estructuras/in_not_diccionarios.out}

Nótese que retorna \ttt{False} al buscar \ttt{"tres"}. Esto ocurre porque \ttt{"tres"} es un valor, no una clave.

\section{Función get()}

Un método útil de diccionarios es \ttt{get()}. Hace lo mismo que indexar, pero si una clave no es encontrada en el diccionario entonces devuelve otro valor especificado (\simple{None} por defecto).

\pythonfile{codigo/16-estructuras/get_1.py}
\out{codigo/16-estructuras/get_1.out}

En este caso es importante recordar que 1 y True son la misma clave. Entonces, el diccionario \doble{pares} sólo tiene 3 elementos (la clave True tendrá como valor False, su último valor).

\pythonfile{codigo/16-estructuras/get_2.py}
\out{codigo/16-estructuras/get_2.out}

%TODO: esta sección

\section{Función keys()}

\section{Tuplas}

Las tuplas son estructuras de datos muy parecidas a las listas, excepto que son inmutables (no pueden ser cambiadas). También se crean utilizando paréntesis en vez de corchetes.

\pythonfile{codigo/16-estructuras/tupla_1.py}

Las tuplas son más rápidas que las listas, pero no pueden ser modificadas.

\section{Indexación de Tuplas}

Se puede acceder a los valores de una tupla utilizando su índice. Funciona de la misma forma que con listas.

\pythonfile{codigo/16-estructuras/tupla_2.py}
\out{codigo/16-estructuras/tupla_2.out}

Los datos dentro de una tupla están ordenados, empezando desde el índice \ttt{0}.

\section{Excepciones en tuplas}

Tratar de reasignar un valor a una tupla ocasiona un \ttt{TypeError}.

\pythonfile{codigo/16-estructuras/error_tupla.py}
\out{codigo/16-estructuras/error_tupla.out}

\section{Anidación de tuplas}

Al igual que las listas y diccionarios, las tuplas pueden ser anidadas entre sí.
Las tuplas son inmutables, pero el contenido de elementos mutables dentro de ellas puede ser cambiado.

\pythonfile{codigo/16-estructuras/anidacion_tuplas.py}
\out{codigo/16-estructuras/anidacion_tuplas.out}

\section{Tupla vacía}

Una tupla vacía se crea utilizando un par de paréntesis vacíos.

\pythonfile{codigo/16-estructuras/tupla_vacia.py}

\section{Manejo de variables con tuplas}

Las tuplas pueden \doble{empaquetarse} o \doble{desempaquetarse}, lo que puede ser útil al momento de crear variables.

\pythonfile{codigo/16-estructuras/tupla_3.py}
\out{codigo/16-estructuras/tupla_3.out}

Las tuplas pueden ser creadas sin paréntesis, simplemente separando los valores por comas.

\pythonfile{codigo/16-estructuras/sin_parentesis.py}
\out{codigo/16-estructuras/sin_parentesis.out}

Se pueden usar tuplas para intercambiar los valores de 2 variables.

\pythonfile{codigo/16-estructuras/truco_tuplas.py}
\out{codigo/16-estructuras/truco_tuplas.out}

\section{Conjuntos}

Los conjuntos son estructuras de datos parecidas a las listas o a los diccionarios. Son creandos utilizando llaves o la función set(). Comparten algunas de las funcionalidades de las listas, como el uso de in para revisar si contienen o no un elemento en particular.

\pythonfile{codigo/16-estructuras/conjuntos.py}
\out{codigo/16-estructuras/conjuntos.out}

No están ordenados, lo cual significa que no pueden ser indexados. No pueden tener elementos duplicados.

\pythonfile{codigo/16-estructuras/sin_duplicados.py}
\out{codigo/16-estructuras/sin_duplicados.out}

Usos básicos de conjuntos incluyen pruebas de membresía y la eliminación de entradas duplicadas.

\section{Conjunto vacío}

Para crear un conjunto vacío, se debe utilizar \ttt{set()}, ya que \ttt{\{\}} crea un diccionario vacío.

\pythonfile{codigo/16-estructuras/conjunto_vacio.py}

\section{Métodos de conjuntos}

Los conjuntos difieren de las listas de varias formas, pero comparten varias operaciones de listas como \ttt{len()}.

\pythonfile{codigo/16-estructuras/len_conjuntos.py}
\out{codigo/16-estructuras/len_conjuntos.out}

Debido a la forma en que son almacenados, es más rápido revisar si un elemento es parte de un conjunto que si es parte de una lista.\smallskip

En lugar de utilizar \ttt{append()} para agregarle algo al conjunto, se utiliza \ttt{add()}. El método \ttt{remove()} elimina un elemento específico de un conjunto.

\pythonfile{codigo/16-estructuras/add_y_remove.py}
\out{codigo/16-estructuras/add_y_remove.out}

El método \ttt{pop()} elimina un elemento arbitrario. Esto significa que debido a la forma en la Python implementa conjuntos, no hay garantía de que los elementos se retornarán en el mismo orden que en el que se añadieron.\smallskip

Generalmente, elimina el primer elemento, pero esto no se puede garantizar.

\pythonfile{codigo/16-estructuras/pop_conjuntos.py}

\section{Operaciones con conjuntos}

Los conjuntos pueden ser combinados utilizando operaciones matemáticas.\smallskip

El operador de unión \ttt{|} combina dos conjuntos para formar uno nuevo que contiene los elementos de cualquiera de los dos.

\pythonfile{codigo/16-estructuras/union.py}
\out{codigo/16-estructuras/union.out}

El operador de intersección \ttt{\&} obtiene sólo los elementos que están en ambos.

\pythonfile{codigo/16-estructuras/interseccion.py}
\out{codigo/16-estructuras/interseccion.out}

El operador de diferencia \ttt{-} obtiene los elementos que están en el primer conjunto, pero no en el segundo.

\pythonfile{codigo/16-estructuras/diferencia.py}
\out{codigo/16-estructuras/diferencia.out}

El operador de diferencia simétrica \ttt{\^} obtiene los elementos que están en cualquiera de los conjuntos, pero no en ambos.

\pythonfile{codigo/16-estructuras/diferencia_simetrica.py}
\out{codigo/16-estructuras/diferencia_simetrica.out}

\section{Resumen sobre estructuras de datos}

Como se ha visto en este capítulo, Python tiene soporte de las siguientes estructuras de datos: listas, diccionarios, tuplas y conjuntos.\smallskip

¿Cúando utilizar diccionarios?

\begin{itemize}
  \item Cuando se necesita utilizar asociaciones lógicas entre pares clave:valor.
  
  \item Cuando se necesita buscar datos rápidamente, en base a claves personalizadas.
  
  \item Cuando los datos son constantemente modificados.
  
\end{itemize}

¿Cúando utilizar listas?

\begin{itemize}
  \item Cuando se tiene un grupo de datos que no necesita acceso aleatorio (deben estar ordenados).
  
  \item Cuando se necesita una recolección simple e iterable que es modificada frecuentemente.
  
\end{itemize}

¿Cúando utilizar conjuntos?

\begin{itemize}
  \item Cuando se necesita que los elementos sean únicos.
\end{itemize}

¿Cúando utilizar tuplas?

\begin{itemize}
  \item Cuando se necesita almacenar datos que no pueden ser cambiados.
\end{itemize}

En muchas ocasiones, una tupla es utilizada junto con un diccionario. Por ejemplo, una tupla puede representar una clave, porque es inmutable.\smallskip

La siguiente tabla muestra las propiedades de cada estructura de datos.\medskip\medskip

% tabla 2

\makebox[\textwidth][c]{
  \begin{tabular}{ |c|c|c|c|c| }
    \hline
    & Lista & Diccionario & Tupla & Conjunto \\
    \hline
    Mutable & \si & \si & \no & \si \\
    \hline
    Indexado & \indices & \claves & \indices & \no \\
    \hline
    Secuencial & \si & \no & \si & \no \\
    \hline
    Elementos únicos & \no & \no & \no & \si \\
    \hline
    Notación & [ ] & \{ \} & ( ) & \{ \} \\
    \hline
  \end{tabular}
}\medskip\medskip

Existen estructuras de datos más complejas, pero no se verán todavía.

\clearpage