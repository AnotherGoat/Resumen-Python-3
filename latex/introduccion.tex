\chapter{Introducción a Python}

\Needspace{8\baselineskip}
\section{¿Qué es Python?}

\link{https://www.python.org}{Python} es un lenguaje de programación popular y de alto nivel.
Fue creado por Gudo van Rossum y fue lanzado en el año 1991.
Se han lanzado diferentes versiones hasta llegar a la versión actual, Python 3.

Según su creador, el nombre de Python viene de la serie de comedia británica \doble{El Circo Volador de Monty Python}.

\Needspace{8\baselineskip}
\section{Aplicaciones}

Python tiene aplicaciones en numerosas áreas y generalmente se usa para:

\begin{itemize}
  \item Scripting
  
  \item Desarrollo web
  
  \item Desarrollo de software
  
  \item Prototipos rápidos
  
  \item Desarrollo de aplicaciones para empresas

  \item Redes y seguridad

  \item Automatización de tareas
  
  \item Programación de bots

  \item Cálculos complejos

  \item Estudios científicos y computación científica
  
  \item Desarrollo de juegos
  
  \item Desarrollo de aplicaciones móviles

  \item Extracción de datos de sitios web

  \item Análisis de datos

  \item Inteligencia artificial y modelos de machine learning

  \item Procesamiento de imágenes

  \item Internet de las cosas
  
\end{itemize}

\Needspace{8\baselineskip}
\section{Componentes principales}

\begin{itemize}
  \item \textbf{Funciones}: Python tiene funciones predefinidas que son de gran utilidad, por ejemplo las funciones matemáticas.
  Las funciones son grupos de bloques de código que pueden ejecutarse en cualquier sección del código, según el programador lo necesite.
  
  \item \textbf{Clases}: Son los planos para construir objetos, los cuales incluyen todas sus características y comportamientos.
  En Python, prácticamente todo es un objeto, aunque en algunos casos no parezca obvio.
  
  \item \textbf{Módulos}: Agrupan funciones y clases que sirven para objetivos similares.
  La biblioteca estándar de Python es muy grande comparada con las de otros lenguajes de programación.
  
  \item \textbf{Paquetes}: Agrupan módulos usados en aplicaciones grandes, permitiendo su distribución a terceros.
  
\end{itemize}

Si estos conceptos no se entienden todavía, no hay problema, pues se verán más a fondo en sus respectivos capítulos.

\Needspace{8\baselineskip}
\section{Características de Python}

\begin{itemize}
  \item \textbf{Soporte de múltiples plataformas}: Python es \doble{platform independent}.
  Esto significa que un mismo código puede ejecutarse en cualquier sistema operativo, ya sea Windows, Unix, Linux o Mac, sin necesidad de hacerle cambios.
  
  \item \textbf{Interpretado}: El código escrito en Python no necesita ser compilado, como ocurre en lenguajes como C, Java o C++.
  El intérprete de Python convierte el código en bytes ejecutables línea por línea, lo cual hace el desarrollo de prototipos más conveniente para el programador, pero al mismo tiempo hace que la ejecución del código sea más lenta.
  
  \item \textbf{Simple}: La sintaxis de Python es simple y fácil de leer, aunque requiere un conocimiento de inglés básico.
  Esta simplicidad también permite escribir código más corto, obteniendo el mismo resultado en menos líneas que en otros lenguajes.
  
  \item \textbf{Robusto}: Esto significa que el lenguaje es capaz de manejar los posibles errores que podrían ocurrir durante la ejecución de programas escritos en él.
  
  \item \textbf{De alto nivel}: Es un lenguaje de nivel usado para scripting.
  Esto quiere decir que el programador no necesita recordar características de bajo nivel como la arquitectura del sistema o el manejo de memoria.
  Estas características se abstraen.
  
  \item \textbf{Multiparadigma}: Python soporta programación estructurada, funcional y orientada a objetos.

  \item \textbf{Gran soporte de librerías}: Python puede integrarse a otras librerías que tengan funcionalidades específicas.
  No hay necesidad de tener que escribir el código por tu cuenta, si ya existe y funciona correctamente.
  
  \item \textbf{Integrable}: El código fuente escrito en Python puede usarse dentro de otros lenguajes de programación, lo cual permite expandir las funcionalidades de sus programas con las de programas escritos en otros lenguajes.
  
  \item \textbf{De código abierto}: Python es de \link{https://opensource.org/about}{código abierto}.
  Se puede usar sin necesidad de tomar su licencias y se puede descargar fácilmente.
  
  \item \textbf{Gratuito}: Ninguna persona u organización necesita pagar para usarlo en sus proyectos.
  
  \item \textbf{Conciso y compacto}: El código escrito en Python es conciso y compacto, lo que sirve para que los programadores puedan entenderlo rápidamente.
  
  \item \textbf{Dinámicamente tipeado}: El tipo de dato de cada variable se decide durante el tiempo de ejecución, lo que significa que no es necesario declarar su tipo en el código.
  
\end{itemize}

\Needspace{8\baselineskip}
\section{Organizaciones que lo usan}

Python es muy popular y es usado por organizaciones y aplicaciones como:

\begin{itemize}
  \item \link{https://www.microsoft.com/en-us/}{Microsoft}

  \item \link{https://www.google.com}{Google}
  
  \item \link{https://www.yahoo.com}{Yahoo}
  
  \item \link{https://www.mozilla.org/en-US/}{Mozilla}
  
  \item \link{https://www.cisco.com}{Cisco}
  
  \item \link{https://www.facebook.com}{Facebook}
  
  \item \link{https://www.spotify.com}{Spotify}
  
  \item \link{https://www.openstack.org}{OpenStack}

  \item \link{https://www.nasa.gov}{NASA}
  
  \item \link{https://www.cia.gov}{CIA}
  
  \item \link{https://www.disney.com}{Disney}
  
\end{itemize}

\Needspace{8\baselineskip}
\section{Ventajas y desventajas}

Ventajas:

\begin{itemize}
  \item Es de código abierto y fácil de empezar a usar.
  
  \item Es fácil de aprender y explorar.
  
  \item Se pueden integrar módulos de terceros con facilidad.
  
  \item Es un lenguaje de alto nivel y orientado a objetos.
  
  \item Es interactivo y portable.
  
  \item Las aplicaciones pueden ejecutarse en cualquier plataforma.
  
  \item Es dinámicamente tipeado.
  
  \item Tiene una comunidad grande y foros activos.
  
  \item Tiene una sintaxis intuitiva.
  
  \item Tiene librerías con amplio soporte.
  
  \item Es un lenguaje interpretado.
  
  \item Se puede conectar con bases de datos.
  
  \item Aumenta la productividad debido a su simplicidad.
  
\end{itemize}

Desventajas:

\begin{itemize}
  \item No puede usarse para desarrollo de aplicaciones móviles.
  
  \item Su acceso a bases de datos es limitado.
  
  \item Consume más memoria por ser dinámicamente tipeado.
  
  \item Su ejecución es lenta comparada con lenguajes compilados.
  
  \item Las aplicaciones y código requieren mayor mantenimiento, debido a su gran grado de abstracción.

  \item Debido a su abstracción, algunos conceptos de programación importantes podrían ser omitidos durante su aprendizaje.

\end{itemize}

\Needspace{8\baselineskip}
\section{Sintaxis}

Python fue diseñado para ser fácil de leer y tiene similitudes con el idioma inglés, con influencia de las matemáticas.

A diferencia de otros lenguajes de programación, Python termina las líneas con un salto de línea, en vez de un punto y coma \ttt{;}.

Para delimitar bloques de código en estructuras de control como \ttt{if}, \ttt{while}, \ttt{for} o funciones, se utiliza indentación con espacios en blanco en vez de usar llaves \ttt{\{\}} como en otros lenguajes.

\Needspace{8\baselineskip}
\section{Instalación}

Muchos PCs vienen con Python preinstalado.
Para revisar la versión que se tiene instalada se puede ingresar el siguiente comando en el terminal.

\textfile{codigo/intro/version.txt}
\out{codigo/intro/version.out}

Siempre se puede descargar la última versión gratuitamente desde el \link{https://www.python.org/downloads/}{sitio web oficial}.
Después de instalarlo, existen muchas formas de utilizar este lenguaje.

Se puede abrir la consola de Python en el terminal ingresando el siguiente comando.

\textfile{codigo/intro/interprete_1.txt}
\out{codigo/intro/interprete_1.out}

Si no funciona, existe otro comando que también puede abrir la consola.

\textfile{codigo/intro/interprete_2.txt}

Dentro de la consola, se puede ingresar el código línea por línea, y es interpretado después de pulsar \ttt{Intro}.
Para salir de la consola, se debe llamar a la función \ttt{exit()}.

\pythonfile{codigo/intro/exit.py}

Para ejecutar un archivo de extensión \ttt{.py}, se puede ingresar \ttt{python} seguido de la ruta relativa (desde el directorio de trabajo) o absoluta del archivo.

\textfile{codigo/intro/ejecutar.txt}

También existen muchos IDEs (Integrated Development Environment) que permiten programar en Python.
Estos contienen muchas herramientas que pueden ser de utilidad.

Algunos IDEs y editores de texto recomendados son los siguientes:

\begin{itemize}
  \item IDLE, que viene preinstalado con Python.
    
  \item \link{https://thonny.org}{Thonny}
  
  \item \link{https://www.jetbrains.com/pycharm/}{Pycharm}

  \item \link{https://code.visualstudio.com}{Visual Studio Code}, que requiere instalar \link{https://marketplace.visualstudio.com/items?itemName=ms-python.python}{esta extensión}.
  
  \item \link{https://www.sublimetext.com}{Sublime Text}
  
\end{itemize}

\Needspace{8\baselineskip}
\section{Hola mundo}

Para mostrar el texto \doble{Hola mundo} en pantalla se puede usar la función \ttt{print()}.

\pythonfile{codigo/intro/hola_mundo_1.py}
\out{codigo/intro/hola_mundo_1.out}

Cada declaración de impresión \ttt{print()} genera texto en una nueva línea.

\pythonfile{codigo/intro/hola_mundo_2.py}
\out{codigo/intro/hola_mundo_2.out}

Si no se le entrega ningún texto a \ttt{print()}, se imprimirá una línea vacía.
El texto puede escribirse entre comillas simples o dobles.

\pythonfile{codigo/intro/hola_mundo_3.py}
\out{codigo/intro/hola_mundo_3.out}

\Needspace{8\baselineskip}
\section{El Zen de Python}

El \link{https://www.python.org/dev/peps/pep-0020/}{Zen de Python} es una colección de 20 \doble{principios}, 19 de ellos escritos por Tim Peters.
Estos principios han servido como inspiración para muchos programadores de todo el mundo a la hora de crear software.

Se mostrará en pantalla como un \doble{huevo de pascua} la primera vez que se ejecute la siguiente linea, para importar el módulo \ttt{this}.

\pythonfile{codigo/intro/import_this.py}

Después se mostrará el siguiente texto, en inglés.

\textfile{codigo/intro/zen.txt}

Una traducción posible sería la siguiente:

\textfile{codigo/intro/zen_spanish.txt}

Nota: Si se observa con claridad, se puede ver que el principio 20 \doble{no existe}.
Tim Peters, quien lo escribió, dejó este último principio para que Guido (el creador de Python) lo llene.
El origen de este texto puede encontrarse \link{https://groups.google.com/g/comp.lang.python/c/B\_VxeTBClM0/m/L8W9KlsiriUJ}{en esta conversación}.

\clearpage