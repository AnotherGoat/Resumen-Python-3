\chapter{Tuplas}

\Needspace{8\baselineskip}
\section{Tuplas}

Las tuplas son estructuras de datos muy parecidas a las listas, excepto que son inmutables (no pueden ser cambiadas).
Se crean utilizando paréntesis en vez de corchetes.

\pythonfile{codigo/tuplas/tupla_1.py}

Las tuplas son más rápidas que las listas, pero no pueden ser modificadas.

\Needspace{8\baselineskip}
\section{Indexación de Tuplas}

Se puede acceder a los valores de una tupla utilizando su índice.
Funciona de la misma forma que con listas.

\pythonfile{codigo/tuplas/tupla_2.py}
\out{codigo/tuplas/tupla_2.out}

Los datos dentro de una tupla están ordenados, empezando desde el índice \ttt{0}.

\Needspace{8\baselineskip}
\section{Excepciones en tuplas}

Tratar de reasignar un valor a una tupla ocasiona un \ttt{TypeError}.

\pythonfile{codigo/tuplas/error_tupla.py}
\out{codigo/tuplas/error_tupla.out}

\Needspace{8\baselineskip}
\section{Anidación de tuplas}

Al igual que las listas y diccionarios, las tuplas pueden ser anidadas entre sí.

Las tuplas son inmutables, pero el contenido de elementos mutables dentro de ellas puede ser cambiado.

\pythonfile{codigo/tuplas/anidacion_tuplas.py}
\out{codigo/tuplas/anidacion_tuplas.out}

\Needspace{8\baselineskip}
\section{Tupla vacía}

Una tupla vacía se crea utilizando un par de paréntesis vacíos \ttt{()}.

\pythonfile{codigo/tuplas/tupla_vacia.py}

\Needspace{8\baselineskip}
\section{Manejo de variables con tuplas}

Las tuplas pueden \doble{empaquetarse} o \doble{desempaquetarse}, lo que puede ser útil al momento de crear variables.

\pythonfile{codigo/tuplas/tupla_3.py}
\out{codigo/tuplas/tupla_3.out}

Las tuplas pueden ser creadas sin paréntesis, simplemente separando los valores por comas.

\pythonfile{codigo/tuplas/sin_parentesis.py}
\out{codigo/tuplas/sin_parentesis.out}

Se pueden usar tuplas para intercambiar los valores de 2 variables, sin necesidad de crear una variable auxiliar.

\pythonfile{codigo/tuplas/truco_tuplas.py}
\out{codigo/tuplas/truco_tuplas.out}

\clearpage