\chapter{Funciones}

\section{Reutilización de código}

La reutilización de código es una parte muy importante de la programación en cualquier lenguaje. Incrementar el tamaño del código lo hace más difícil de mantener.\smallskip

Para que un proyecto grande de programación sea exitoso, es esencial que se atenga al principio DRY, del inglés Don't Repeat Yourself (no te repitas).\smallskip

Del código malo y repetitivo, se dice que se atiene al principio WET, del inglés Write Everything Twice (escribe todo dos veces) o We Enjoy Typing (disfrutamos escribir).\smallskip

Los bucles vistos en el capítulo anterior son una forma de reutilizar código. Este capítulo trata sobre funciones, que son otra forma de hacer esta práctica.

\section{Funciones}

Cualquier sentencia que consista de una palabra seguida de información entre paréntesis es llamada una función.\smallskip

Ejemplos de funciones que se han visto anteriormente:

\pythonfile{codigo/8-funciones/funciones.py}

Palabras antes del paréntesis son nombres de funciones y los valores separados por comas dentro de los paréntesis son argumentos o parámetros de funciones.

\section{Definición de funciones}

Para definir una función, se debe usar la palabra clave \ttt{def}, seguida del nombre de la función y de un paréntesis que puede o no incluir parámetros. El cuerpo de la función incluye el código y debe tener un grado de indentación mayor que el de la definición.\smallskip

Generalmente, se recomienda que el nombre de una función sea un verbo, pero no es obligatorio.\smallskip

Ejemplo de definición de función:

\pythonfile{codigo/8-funciones/saludar.py}

\section{Argumentos}

Las funciones más básicas son las que no reciben argumentos, pero esto limita su utilidad en gran cantidad. Para hacerlas más útiles, se les puede entregar argumentos, lo que aumenta su funcionalidad.

\pythonfile{codigo/8-funciones/exclamar.py}

Una función puede pedir más de un argumento, los cuales deben separarse por comas.

\pythonfile{codigo/8-funciones/sumar.py}

\pythonfile{codigo/8-funciones/mayor.py}

Los argumentos de funciones pueden ser utilizados como variables dentro de la definición de la función. Sin embargo, no pueden ser referenciados fuera de la definición de la función. Esto también se aplica a las demás variables creadas dentro de una función.

\pythonfile{codigo/8-funciones/argumentos.py}
\out{codigo/8-funciones/argumentos.out}

\section{Llamado de funciones}

Para llamar una función, debe escribirse su nombre, seguido de un paréntesis que contiene los argumentos (información) que se le quieren entregar.\smallskip

Ejemplos de uso de las funciones definidas anteriormente:

\pythonfile{codigo/8-funciones/llamar_saludar.py}
\out{codigo/8-funciones/llamar_saludar.out}

\pythonfile{codigo/8-funciones/llamar_exclamar.py}
\out{codigo/8-funciones/llamar_exclamar.out}

\pythonfile{codigo/8-funciones/llamar_sumar.py}
\out{codigo/8-funciones/llamar_sumar.out}

\pythonfile{codigo/8-funciones/llamar_mayor.py}
\out{codigo/8-funciones/llamar_mayor.out}

Técnicamente, los parámetros son las variables en una definición de función y los argumentos son los valores que se le dan a las variables al momento de llamar funciones.

\section{Devolución de valores en una función}

Ciertas funciones devuelven un valor para ser utilizado más adelante. Para hacer esto en la definición de funciones nuevas, se debe usar la palabra clave \ttt{return}.

\pythonfile{codigo/8-funciones/return_1.py}
\out{codigo/8-funciones/return_1.out}

Cualquier código luego de la sentencia \ttt{return} nunca ocurrirá.

\pythonfile{codigo/8-funciones/return_2.py}
\out{codigo/8-funciones/return_2.out}

\section{Docstring}

Las docstring (cadenas de documentación) cumplen un propósito similar al de los comentarios, pero son más específicos y tienen una sintaxis distinta.\smallskip

Son creados colocando una cadena multilínea que contenga una explicación de la función por debajo de la primera línea de la función y teniendo cuidado de respetar la indentación.

\pythonfile{codigo/8-funciones/docstring_1.py}

Normalmente se usan docstrings para explicar los parámetros y lo que retorna una función.

%TODO: buscar formato aceptado
\pythonfile{codigo/8-funciones/docstring_2.py}

El intérprete de Python ignora los docstring que se usen en otros lugares. Esto ha hecho que algunas personas los usen como \doble{comentarios multilínea}, pero en la práctica no se recomienda hacer esto porque puede generar confusión.

\pythonfile{codigo/8-funciones/comentario_malo.py}

También podría generar problemas de indentación en algunos casos.

\section{Revisar documentación} %TODO: revisar más info aquí

A diferencia de los comentarios convencionales, los docstring se conservan a lo largo del tiempo de ejecución del programa. Esto le permite al programador examinar estos comentarios durante el tiempo de ejecución.\smallskip

Una forma de accederlo es usando la variable \ttt{\_\_doc\_\_}.

\pythonfile{codigo/8-funciones/print_doc.py}
\out{codigo/8-funciones/print_doc.out}

Otra forma es usando la función \ttt{help()}. Esto entrega un poco más de información.

\pythonfile{codigo/8-funciones/help.py}
\out{codigo/8-funciones/help.out}

\section{Funciones como objetos}

Aunque sean creadas de manera diferente que las variables regulares, las funciones son como cualquier otra clase de valor. Pueden ser asignadas y reasignadas a variables, y luego ser referenciadas por esos nombres.

\pythonfile{codigo/8-funciones/funcion_como_objeto_1.py}
\out{codigo/8-funciones/funcion_como_objeto_1.out}

Las funciones también pueden ser usadas como argumentos de otras funciones.

\pythonfile{codigo/8-funciones/funcion_como_objeto_2.py}
\out{codigo/8-funciones/funcion_como_objeto_2.out}

\section{El objeto None}

El objeto \ttt{None} es utilizado para representar la ausencia de un valor. Es similar a \ttt{null} en otros lenguajes de programación.\smallskip

Al igual que otros valores \doble{vacíos}, tales como \ttt{()}, \ttt{[]}, \ttt{\{\}} y la cadena vacía \ttt{\qq \qq} o \ttt{\q \q}, es \ttt{False} cuando es convertido a una variable booleana.

%TODO: ejemplo y mencionar estructuras de datos

Cuando es ingresado a la consola de Python, se visualiza como una cadena vacía. Usando \ttt{print()} se mostrará su verdadero valor.

\pythonfile{codigo/8-funciones/None_1.py}
\out{codigo/8-funciones/None_1.out}

El objeto \ttt{None} es devuelto por cualquier función que no devuelve explícitamente algo más.

\pythonfile{codigo/8-funciones/None_2.py}
\out{codigo/8-funciones/None_2.out}

Un uso común de \ttt{None} es para iniciar una lista con cierta cantidad de elementos.

\pythonfile{codigo/8-funciones/None_3.py}
\out{codigo/8-funciones/None_3.out}

\section{Sobrecarga de funciones}

En Python, la sobrecarga de funciones se define como la habilidad de que una función se comporte de distinta manera dependiendo del número de argumentos que reciba. Esto permite reutilizar el mismo código, reducir su complejidad y hacer que sea más fácil de leer.\smallskip

Para sobrecargar una función, se deben definir los parámetros que sean opcionales con un valor por defecto \ttt{None}. Después, se usan declaraciones \ttt{if-elif-else} para revisar todos los casos posibles.

\pythonfile{codigo/8-funciones/sobrecarga_1.py}
\out{codigo/8-funciones/sobrecarga_1.out}

El siguiente ejemplo muestra una función que calcula el área de una figura geométrica. Si sólo se le entrega 1 argumento asume que es un cuadrado y si se le entregan 2 asume que es un rectángulo.

\pythonfile{codigo/8-funciones/sobrecarga_2.py}
\out{codigo/8-funciones/sobrecarga_2.out}

%TODO: mostrar valores por defecto

\section{Anotaciones de tipos}

Una anotación de tipos es, como su nombre lo dice, una notación opcional que especifica el tipo de los parámetros de una función y su tipo de retorno.

\pythonfile{codigo/8-funciones/anotaciones_1.py}
\out{codigo/8-funciones/anotaciones_1.out}

En el ejemplo de arriba, se muestra que mensaje es un parámetro de tipo string y que la función \ttt{duplicar()} retorna un string.\smallskip

Esto le permite al programador saber qué tipos de datos se le deben entregar una función y que tipos de datos esperar cuando esta función retorne.\smallskip

Las anotaciones de tipos son ignoradas completamente por el intérprete de Python. No restringen el tipo de los parámetros o del retorno de una función, pero son muy útiles al momento de documentar.

\pythonfile{codigo/8-funciones/anotaciones_2.py}
\out{codigo/8-funciones/anotaciones_2.out}

Aunque el intérprete de Python los ignore, existen algunos IDEs y programas que pueden analizar código que contiene anotaciones de tipos y alertar sobre problemas potenciales.

\clearpage