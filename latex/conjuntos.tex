\chapter{Conjuntos}

\Needspace{8\baselineskip}
\section{Conjuntos}

Los conjuntos son estructuras de datos parecidas a las listas o a los diccionarios.
Son creandos utilizando llaves \ttt{\{\}}o la función \ttt{set()}.
Comparten algunas de las funcionalidades de las listas, como el uso de \ttt{in} y \ttt{not in} para revisar si contienen o no un elemento en particular.

\pythonfile{codigo/conjuntos/conjuntos.py}
\out{codigo/conjuntos/conjuntos.out}

No están ordenados, lo cual significa que no pueden ser indexados.
No pueden tener elementos duplicados.

\pythonfile{codigo/conjuntos/sin_duplicados.py}
\out{codigo/conjuntos/sin_duplicados.out}

Usos básicos de conjuntos incluyen pruebas de membresía y la eliminación de entradas duplicadas.

\Needspace{8\baselineskip}
\section{Conjunto vacío}

Para crear un conjunto vacío, se debe utilizar \ttt{set()}, ya que \ttt{\{\}} crea un diccionario vacío.

\pythonfile{codigo/conjuntos/conjunto_vacio.py}

\Needspace{8\baselineskip}
\section{Métodos de conjuntos}

Los conjuntos difieren de las listas de varias formas, pero comparten varias operaciones de listas como \ttt{len()}.

\pythonfile{codigo/conjuntos/len_conjuntos.py}
\out{codigo/conjuntos/len_conjuntos.out}

Debido a la forma en que son almacenados, es más rápido revisar si un elemento es parte de un conjunto que si es parte de una lista.

En lugar de utilizar \ttt{append()} para agregarle algo al conjunto, se utiliza \ttt{add()}.
El método \ttt{remove()} elimina un elemento específico de un conjunto.

\pythonfile{codigo/conjuntos/add_y_remove.py}
\out{codigo/conjuntos/add_y_remove.out}

El método \ttt{pop()} elimina un elemento arbitrario.
Esto significa que debido a la forma en la Python implementa conjuntos, no hay garantía de que los elementos se retornarán en el mismo orden que en el que se añadieron.

Generalmente, elimina el primer elemento, pero esto no se puede garantizar para todos los casos.

\pythonfile{codigo/conjuntos/pop_conjuntos.py}

\Needspace{8\baselineskip}
\section{Operaciones con conjuntos}

Los conjuntos pueden ser combinados utilizando operaciones matemáticas.

El operador de unión \ttt{|} combina dos conjuntos para formar uno nuevo que contiene los elementos de cualquiera de los dos.

\pythonfile{codigo/conjuntos/union.py}
\out{codigo/conjuntos/union.out}

El operador de intersección \ttt{\&} obtiene sólo los elementos que están en ambos.

\pythonfile{codigo/conjuntos/interseccion.py}
\out{codigo/conjuntos/interseccion.out}

El operador de diferencia \ttt{-} obtiene los elementos que están en el primer conjunto, pero no en el segundo.

\pythonfile{codigo/conjuntos/diferencia.py}
\out{codigo/conjuntos/diferencia.out}

El operador de diferencia simétrica \ttt{\^} obtiene los elementos que están en cualquiera de los conjuntos, pero no en ambos.

\pythonfile{codigo/conjuntos/diferencia_simetrica.py}
\out{codigo/conjuntos/diferencia_simetrica.out}

\Needspace{8\baselineskip}
\section{Frozenset}

Los frozenset son, como dice su nombre, conjuntos \doble{congelados}.
Se comportan exactamente igual que un conjunto normal, pero son inmutables.
Para crearlos, se debe usar la función \ttt{frozenset()}.

\pythonfile{codigo/conjuntos/frozenset.py}
\out{codigo/conjuntos/frozenset.out}

Como son inmutables, intentar cambiarlos causará una excepción.
De hecho, ni siquiera existen métodos para modificarlos.

\pythonfile{codigo/conjuntos/error_frozenset_1.py}
\out{codigo/conjuntos/error_frozenset_1.out}
\pythonfile{codigo/conjuntos/error_frozenset_2.py}
\out{codigo/conjuntos/error_frozenset_2.out}
\pythonfile{codigo/conjuntos/error_frozenset_3.py}
\out{codigo/conjuntos/error_frozenset_3.out}
\pythonfile{codigo/conjuntos/error_frozenset_4.py}
\out{codigo/conjuntos/error_frozenset_4.out}

\Needspace{8\baselineskip}
\section{Resumen sobre estructuras de datos}

Como se ha visto en los capítulos anteriores, Python tiene soporte de las siguientes estructuras de datos: listas, diccionarios, tuplas y conjuntos.

¿Cúando utilizar diccionarios?

\begin{itemize}
  \item Cuando se necesita utilizar asociaciones lógicas entre pares clave:valor.
  
  \item Cuando se necesita buscar datos rápidamente, en base a claves personalizadas.
  
  \item Cuando los datos son constantemente modificados.
  
\end{itemize}

¿Cúando utilizar listas?

\begin{itemize}
  \item Cuando se tiene un grupo de datos que no necesita acceso aleatorio (deben estar ordenados).
  
  \item Cuando se necesita una recolección simple e iterable que es modificada frecuentemente.
  
\end{itemize}

¿Cúando utilizar conjuntos?

\begin{itemize}
  \item Cuando se necesita que los elementos sean únicos.
\end{itemize}

¿Cúando utilizar tuplas?

\begin{itemize}
  \item Cuando se necesita almacenar datos que no pueden ser cambiados.
\end{itemize}

En muchas ocasiones, una tupla es utilizada junto con un diccionario.
Por ejemplo, una tupla puede representar una clave, porque es inmutable.

La siguiente tabla muestra las propiedades de cada estructura de datos vista hasta ahora.
También se muestran cadenas ya que tienen propiedades similares.
\medskip\medskip

% tabla 2

\makebox[\textwidth][c]{
  \begin{tabular}{ l l l l l l l }
    \toprule
    & Cadena & Lista & Diccionario & Tupla & Conjunto & Frozenset \\
    \midrule
    Mutable & \no & \si & \si & \no & \si & \no \\

    Indexado & \indices & \indices & \claves & \indices & \no & \no \\

    Secuencial & \si & \si & \no & \si & \no & \no \\

    Elementos únicos & \no & \no & \no & \no & \si & \si \\

    Tipos de datos & \caracteres & \cualquier & \cualquier & \cualquier & \cualquier & \cualquier \\

    Notación & \ttt{\q \q} o \ttt{\qq \qq} & \ttt{[]} & \ttt{\{:\}} & \ttt{()} & \ttt{\{\}} o \ttt{set()} & \ttt{frozenset()} \\
    \bottomrule
  \end{tabular}
}
\medskip\medskip

Existen estructuras de datos más complejas, pero no se verán todavía.

\clearpage