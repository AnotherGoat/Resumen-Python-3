\chapter{Variables}

\Needspace{8\baselineskip}
\section{Asignación de variables}

Una variable permite almacenar un valor asignándole un nombre, el cual puede ser usado para referirse al valor más adelante en el programa.
Para asignar una variable, se usa el signo de igualdad \ttt{=}.

\pythonfile{codigo/4-variables/variable_1.py}
\out{codigo/4-variables/variable_1.py}

En Python, no existe la declaración de variables.
Toda variable que se cree debe tener un valor inicial asignado.
Se permite que este valor pueda ser \doble{vacío}, como por ejemplo el string vacío \ttt{\qq \qq}.

\Needspace{8\baselineskip}
\section{Nombre de variables válidos}

Se aplican ciertas restricciones con respecto a los caracteres que se pueden usar en los nombres de variables.
Los únicos caracteres permitidos son letras, números y guiones bajos \ttt{\_}.
Además, no se puede comenzar con números o incluir espacios.

No seguir estas reglas hará que ocurran errores que impiden la ejecución del programa.

\pythonfile{codigo/4-variables/error_variable_1.py}
\out{codigo/4-variables/error_variable_1.out}
\pythonfile{codigo/4-variables/error_variable_2.py}
\out{codigo/4-variables/error_variable_2.out}
\pythonfile{codigo/4-variables/error_variable_3.py}
\out{codigo/4-variables/error_variable_3.out}

Python es sensible a mayúsculas y minúsculas, lo que significa que las variables \ttt{num}, \ttt{Num}, \ttt{NUM}, etc. son distintas.

\pythonfile{codigo/4-variables/mayusculas_y_minusculas.py}
\out{codigo/4-variables/mayusculas_y_minusculas.out}

\Needspace{8\baselineskip}
\section{Nombres largos de variables}

Los nombres de variables que se componen de más de una palabra pueden ser difíciles de leer.
Existen muchas técnicas para escribirlos de forma más legible.

\begin{itemize}
  \item \textbf{Camel Case}: Todas las palabras empiezan con mayúscula, excepto la primera.
  
  \pythonfile{codigo/4-variables/camelCase.py}

  \item \textbf{Pascal Case}: Todas las palabras empiezan con mayúscula.
  
  \pythonfile{codigo/4-variables/PascalCase.py}

  \item \textbf{Snake Case}: Todas las palabras se escriben en minúscula, pero van separadas por guiones bajos \ttt{\_}.
  
  \pythonfile{codigo/4-variables/snake_case.py}

\end{itemize}

\Needspace{8\baselineskip}
\section{Palabras clave}

Existen palabras específicas que tampoco se pueden usar como nombres de variables.
El intérprete de Python reconoce estas palabras como palabras clave o keywords, y tienen usos reservados.

A continuación, se muestra una lista de todas las palabras clave en Python.

\textfile{codigo/4-variables/palabras_clave.txt}

Nótese el uso de mayúsculas al principio de \ttt{False}, \ttt{None} y \ttt{True}.

\Needspace{8\baselineskip}
\section{Operaciones con variables}

Se pueden usar variables dentro de operaciones.

\pythonfile{codigo/4-variables/variable_2.py}
\out{codigo/4-variables/variable_2.out}

Pero se debe recordar que deben ser asignadas antes de poder usarlas.

\pythonfile{codigo/4-variables/error_operacion.py}
\out{codigo/4-variables/error_operacion.out}

Una variable también puede cambiar de valor a lo largo de la ejecución de un programa, y ser el resultado de operaciones con otras variables.

\pythonfile{codigo/4-variables/variable_3.py}

Durante el tiempo de ejecución, siempre mantendrá el último valor que se le asignó.

\pythonfile{codigo/4-variables/conservacion.py}
\out{codigo/4-variables/conservacion.out}

Las variables se pueden reasignar tantas veces como se desee, para cambiar su valor.
En Python, las variables no tienen tipos específicos, por lo que se puede asignar una cadena a una variable y luego asignar un número entero a la misma variable.

\pythonfile{codigo/4-variables/variable_4.py}
\out{codigo/4-variables/variable_4.out}

Sin embargo, no es una buena práctica.
Para evitar errores, se debería evitar sobreescribir una variable con distintos tipos de datos.

%TODO: enlaces a numérico y alfanumérico
Si se asigna el valor numérico (enteros y decimales) o alfanumérico (cadena de texto) de una variable a otra, los cambios hechos a ambas variables son independientes.

\pythonfile{codigo/4-variables/variable_5.py}
\out{codigo/4-variables/variable_5.out}

Esto no ocurre para tipos de datos más complejos, como listas.

\Needspace{8\baselineskip}
\section{Entrada}

Para obtener información del usuario, se puede usar la función \ttt{input()}, la cual solicita al usuario que la entrada.
La información obtenida puede ser almacenada como una variable.

\pythonfile{codigo/4-variables/input_1.py}
\inp{codigo/4-variables/input_1.inp}
\out{codigo/4-variables/input_1.out}

Toda la información recibida por el método \ttt{input()} se procesa como un string.

También se puede entregar un string como parámetro al método \ttt{input()}, lo cual mostrará texto antes de pedir la entrada.
Esto sirve para aclarar qué entrada está solicitando el programa.

\pythonfile{codigo/4-variables/input_2.py}

Al usar la función \ttt{input()}, el flujo del programa se detiene hasta que el usuario ingrese algún valor.

Se puede usar \ttt{input()} varias veces para tomar múltiples entradas del usuario.

\pythonfile{codigo/4-variables/input_multiple.py}
\inp{codigo/4-variables/input_multiple.inp}
\out{codigo/4-variables/input_multiple.out}

Normalmente, se debe añadir un espacio para concatenar variables, para evitar que las variables aparezcan juntas en la salida.

\Needspace{8\baselineskip}
\section{Conversión de tipos de datos}

Existen funciones para convertir datos de un tipo a otro.
Entre dichas funciones se encuentran:

\begin{itemize}
  \item \ttt{int()}: Convierte a un número entero.
  
  \pythonfile{codigo/4-variables/convertir_a_int_1.py}
  \out{codigo/4-variables/convertir_a_int_1.out}

  Aunque ambos se impriman igual, el primero es un string y el segundo un entero (int).

  \pythonfile{codigo/4-variables/convertir_a_int_2.py}
  \out{codigo/4-variables/convertir_a_int_2.out}

  Al convertir de \ttt{float} a \ttt{int}, sólo se conservará la parte entera del número.
  No ocurrirá ningún tipo de redondeo.

  \pythonfile{codigo/4-variables/float_a_int.py}
  \out{codigo/4-variables/float_a_int.out}

  \item \ttt{float()}: Convierte a un float.
  
  \pythonfile{codigo/4-variables/convertir_a_float.py}
  \out{codigo/4-variables/convertir_a_float.out}

  \item \ttt{str()}: Convierte a un string o cadena de caracteres.
  Uno de sus usos principales es para concatenar distintos tipos de datos.

  \pythonfile{codigo/4-variables/convertir_a_string.py}
  \out{codigo/4-variables/convertir_a_string.out}

\end{itemize}

Como el resultado de la función \ttt{input()} es un string, es normal convertir el resultado de este método para usarlo en operaciones matemáticas.

\pythonfile{codigo/4-variables/input_3.py}
\inp{codigo/4-variables/input_3.inp}
\out{codigo/4-variables/input_3.out}

También se pueden usar estos métodos para forzar variables a ser de un tipo específico al momento de asignarlas, en vez de dejar que Python infiera su tipo.
Esto se conoce como \doble{casting}.

\pythonfile{codigo/4-variables/casting.py}

\Needspace{8\baselineskip}
\section{Operadores de asignación}

Los operadores de asignación permiten escribir código como \ttt{x = x + 1} de manera más concisa, como \ttt{x += 1}.
Lo mismo es posible con otros operadores como \ttt{-}, \ttt{*}, \ttt{/}, \ttt{//}, \ttt{\%} y \ttt{**}.

\pythonfile{codigo/4-variables/asignacion_1.py}

También se pueden usar con los operadores de concatenación y multiplicación de strings.

\pythonfile{codigo/4-variables/asignacion_2.py}

\Needspace{8\baselineskip}
\section{Múltiples variables con el mismo valor}

Para declarar muchas variables en una misma linea y hacer que todas tengan el mismo valor, se puede repetir el operador de asignación \ttt{=}.

\pythonfile{codigo/4-variables/multiple.py}
\out{codigo/4-variables/multiple.out}

\Needspace{8\baselineskip}
\section{Tipos de datos}

En Python existen muchos tipos de datos que pueden ser usados y almacenados como variables.
La función \ttt{type()} retorna el tipo de dato del objeto que se le entrega como argumento.

A continuación se mostrarán los tipos de datos más comunes.

\begin{itemize}
  \item Número entero \ttt{int}
  
  \pythonfile{codigo/4-variables/tipo_int.py}
  \out{codigo/4-variables/tipo_int.out}

  \item Número racional \ttt{float}
  
  \pythonfile{codigo/4-variables/tipo_float.py}
  \out{codigo/4-variables/tipo_float.out}

  \item Cadena de caracteres \ttt{str}
  
  \pythonfile{codigo/4-variables/tipo_str.py}
  \out{codigo/4-variables/tipo_str.out}

  \item Booleano \ttt{bool}
  
  \pythonfile{codigo/4-variables/tipo_bool.py}
  \out{codigo/4-variables/tipo_bool.out}

  \item Lista \ttt{list}
  
  \pythonfile{codigo/4-variables/tipo_list.py}
  \out{codigo/4-variables/tipo_list.out}

  \item Rango \ttt{range}
  
  \pythonfile{codigo/4-variables/tipo_range.py}
  \out{codigo/4-variables/tipo_range.out}

  \item Función \ttt{function}
  
  \pythonfile{codigo/4-variables/tipo_function.py}
  \out{codigo/4-variables/tipo_function.out}

  \item Número complejo \ttt{complex}
  
  \pythonfile{codigo/4-variables/tipo_complex.py}
  \out{codigo/4-variables/tipo_complex.out}

  \item Diccionario \ttt{dict}
  
  \pythonfile{codigo/4-variables/tipo_dict.py}
  \out{codigo/4-variables/tipo_dict.out}

  \item Tupla \ttt{tuple}
  
  \pythonfile{codigo/4-variables/tipo_tuple.py}
  \out{codigo/4-variables/tipo_tuple.out}

  \item Conjunto \ttt{set}
  
  \pythonfile{codigo/4-variables/tipo_set.py}
  \out{codigo/4-variables/tipo_set.out}

  \item Frozenset \ttt{frozenset}
  
  \pythonfile{codigo/4-variables/tipo_frozenset.py}
  \out{codigo/4-variables/tipo_frozenset.out}

\end{itemize}

Estos tipos y muchos más se verán en detalle en los próximos capítulos.

\clearpage