\chapter{El módulo itertools}

\Needspace{8\baselineskip}
\section{El módulo itertools}

El módulo itertools es una biblioteca estándar que contiene varias funciones que son útiles en la programación funcional.

\pythonfile{codigo/19-itertools/itertools.py}

\Needspace{8\baselineskip}
\section{Iteradores infinitos}

Uno de los tipos de función que produce son iteradores infinitos.

La función \ttt{count()} cuenta infinitamente a partir de un valor.

\pythonfile{codigo/19-itertools/count_1.py}
\out{codigo/19-itertools/count_1.out}

Se le puede entregar un segundo parámetro a \ttt{count()}, el cual representa el tamaño de paso.
Si no se le entrega el segundo parámetro, el tamaño de paso por defecto es 1.

\pythonfile{codigo/19-itertools/count_2.py}
\out{codigo/19-itertools/count_2.out}

Si el segundo parámetro es negativo, cuenta hacia atrás.

\pythonfile{codigo/19-itertools/count_3.py}
\out{codigo/19-itertools/count_3.out}

La función\ttt{cycle()} itera infinitamente a través de un iterable (como una lista o cadena).

\pythonfile{codigo/19-itertools/cycle_1.py}
\out{codigo/19-itertools/cycle_1.out}
\pythonfile{codigo/19-itertools/cycle_2.py}
\out{codigo/19-itertools/cycle_2.out}

La función \ttt{repeat()} repite un objeto, ya sea infinitamente o un número específico de veces.

\pythonfile{codigo/19-itertools/repeat_1.py}
\out{codigo/19-itertools/repeat_1.out}

Puede usarse para crear listas que tengan un objeto repetido.

\pythonfile{codigo/19-itertools/repeat_2.py}
\out{codigo/19-itertools/repeat_2.out}

Para repetir infinitamente, se le debe entregar sólo el objeto como parámetro.

\pythonfile{codigo/19-itertools/repeat_3.py}
\out{codigo/19-itertools/repeat_3.out}

\Needspace{8\baselineskip}
\section{Operaciones sobre iterables}

Hay muchas funciones en itertools que operan sobre iterables, de una forma similar a \ttt{map()} o \ttt{filter()}.

La función \ttt{accumulate()} devuelve un total actualizado de los valores dentro de un iterable.

\pythonfile{codigo/19-itertools/accumulate.py}
\out{codigo/19-itertools/accumulate.out}

La función \ttt{takewhile()} toma elementos de un iterable mientras una función predicado permanece verdadera.

\pythonfile{codigo/19-itertools/takewhile_1.py}
\out{codigo/19-itertools/takewhile_1.out}

La diferencia entre \ttt{takewhile()} y \ttt{filter()} es que \ttt{takewhile()} deja de tomar elementos cuando llega al primer elemento que no cumple la condición de la función.

\pythonfile{codigo/19-itertools/takewhile_2.py}
\out{codigo/19-itertools/takewhile_2.out}

La función \ttt{chain()} combina varios iterables en uno solo más largo.

\pythonfile{codigo/19-itertools/chain.py}
\out{codigo/19-itertools/chain.out}

\Needspace{8\baselineskip}
\section{Funciones de combinatoria}

También hay numerosas funciones combinatorias en itertools, tales como \ttt{product()} y \ttt{permutations()}.
Estas son usadas cuando se quiere cumplir tareas con todas las combinaciones posibles de algunos elementos.

La función \ttt{product()} retorna el producto entre 2 iterables.

\pythonfile{codigo/19-itertools/product.py}
\out{codigo/19-itertools/product.out}

La función \ttt{permutations()} retorna las todas permutaciones posibles entre los elementos de un iterable.

\pythonfile{codigo/19-itertools/permutations_1.py}
\out{codigo/19-itertools/permutations_1.out}
\pythonfile{codigo/19-itertools/permutations_2.py}
\out{codigo/19-itertools/permutations_2.out}

Se le puede entregar un segundo argumento, el cual representa el número de elementos que cada permutación debe tener.

\pythonfile{codigo/19-itertools/permutations_3.py}
\out{codigo/19-itertools/permutations_3.out}

El método \ttt{combinations()} funciona de forma similar a \ttt{permutations()}, pero muestra las combinaciones en vez de permutaciones.

\pythonfile{codigo/19-itertools/combinations.py}
\out{codigo/19-itertools/combinations.out}

La diferencia es que en las combinaciones no importa el orden y se unen todos los resultados que tienen los mismos elementos pero en orden distinto.

\clearpage