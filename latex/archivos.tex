\chapter{Archivos}

%TODO: revisión de \ttt{} lista hasta aquí

\section{Abrir archivos}

Python se puede usar para leer y escribir los contenidos de archivos. Los archivos de texto son los más fáciles de manipular.\smallskip

Antes de que un archivo pueda ser editado, debe ser abierto con la función open().

\pythonfile{codigo/14-archivos/open.py}

El argumento de la función open es la ruta del archivo. Esta ruta puede ser \link{https://www.discoduroderoer.es/rutas-relativas-y-absolutas/}{relativa} (desde el directorio de trabajo) o \link{https://www.discoduroderoer.es/rutas-relativas-y-absolutas/}{absoluta}.\smallskip

Si se necesita, se puede hacer \link{https://stackoverflow.com/questions/7165749/open-file-in-a-relative-location-in-python}{un procedimiento un poco más largo} para abrir un archivo relativamente desde el directorio donde se encuentra el archivo \ttt{.py}, sin importar cual sea el directorio de trabajo o sistema operativo.

\pythonfile{codigo/14-archivos/open_seguro.py}

En este caso, \ttt{ruta\_relativa} es lo único que debe cambiarse para abrir un archivo distinto dentro del mismo directorio.

\section{Modos de apertura}

Se puede especificar el modo utilizado para abrir un archivo al pasar un segundo argumento a la función open().

Los modos de apertura son:

\begin{itemize}
  \item \ttt{''r''}: Significa modo de lectura, el cual es el modo predeterminado.
  
  \pythonfile{codigo/14-archivos/modo_r.py}

  \item \ttt{''w''}: Significa modo de escritura, el cual sirve para reescribir los contenidos de un archivo. Abrir un archivo en este modo inmediatamente borra todos sus contenidos.
  
  \pythonfile{codigo/14-archivos/modo_w.py}

  \item \ttt{''x''}: Significa modo de creación exclusiva, se usa para crear un fichero y escribir sobre él.
  
  \pythonfile{codigo/14-archivos/modo_x.py}

  Entrega un error FileExistsError si el archivo ya existe y no está vacío.

  \pythonfile{codigo/14-archivos/error_modo_x.py}
  \Ainicial{codigo/14-archivos/error_modo_x.txt_i}
  \out{codigo/14-archivos/error_modo_x.out}

  \item \ttt{''a''}: significa modo de anexo, para agregar nuevo contenido al final de un archivo.

  \pythonfile{codigo/14-archivos/modo_a.py}

\end{itemize}

Además de los modos de apertura, también existen los modos en los que se muestra la información:

\begin{itemize}
  \item \ttt{''t''}: significa abrir el archivo en modo texto (predeterminado). No es necesario escribirlo.
  
  \item \ttt{''b''}: significa abrir el archivo en modo binario, que es utilizado para archivos que no son de texto (tales como archivos de imágenes o sonido). Se debe combinar con alguno de los modos anteriores.

  \pythonfile{codigo/14-archivos/modo_b.py}

\end{itemize}

\section{Extensión de modos de apertura}

Se puede añadir el signo \ttt{''+''} a cualquiera de los modos de arriba para darles acceso adicional a archivos.\smallskip

La siguiente tabla muestra el funcionamiento de cada modo:\medskip

% tabla 1

\makebox[\textwidth][c]{
  \begin{tabular}{ |c|c|c|c|c|c|c|c|c| }
    \hline
    & r & r+ & w & w+ & x & x+ & a & a+ \\
    \hline
    Lee el archivo & \si & \si & \no & \si & \no & \si & \no & \si \\
    \hline
    Escribe en el archivo & \no & \si & \si & \si & \si & \si & \si & \si \\
    \hline
    Crea el archivo si no existe & \no & \no & \si & \si & \si & \si & \si & \si \\
    \hline
    Borra todos los contenidos del archivo & \no & \no & \si & \si & \noo & \noo & \no & \no \\
    \hline
    Posición del cursor & \inicio & \inicio & \inicio & \inicio & \inicio & \inicio & \final & \final \\
    \hline
  \end{tabular}
}\medskip

\textcolor{Crimson}{*}: Levanta una excepción FileExistsError si el archivo ya tiene contenido.

\section{Cierre de archivos}

Una vez que un archivo haya sido abierto y utilizado, es un buen hábito cerrarlo. Esto se logra con el método \ttt{close()} de un objeto archivo.

\pythonfile{codigo/14-archivos/close.py}

A nivel de sistemas operativos, cada proceso tiene un límite en la cantidad de archivos que puede abrir simultáneamente. Si un programa se ejecuta por mucho tiempo y podría abrir muchos archivos, se debe tener cuidado.

\section{Lectura de archivos}

Los contenidos de un archivo que ha sido abierto en modo texto pueden ser leídos utilizando el método \ttt{read()}.

\pythonfile{codigo/14-archivos/read_1.py}
\Ainicial{codigo/14-archivos/read_1.txt_i}
\out{codigo/14-archivos/read_1.out}

Para leer sólo una determinada parte de un archivo, se puede proveer un número como argumento a la función \ttt{read()}. Esto determina el número de bytes que deberían ser leídos.\smallskip

Se pueden hacer más llamadas a \ttt{read()} en el mismo objeto archivo para leer más de él byte por byte. Si no se le pasan argumentos, o si el argumento es negativo, \ttt{read()} devuelve el resto del archivo.

\pythonfile{codigo/14-archivos/read_2.py}
\Ainicial{codigo/14-archivos/read_2.txt_i}
\out{codigo/14-archivos/read_2.out}

El ejemplo de arriba primero mostrará los bytes del 1 al 16, después del 17 al 24, del 25 al 28, y del 29 al último byte.\smallskip

Luego de que todos los contenidos de un archivo hayan sido leídos, cualquier intento de leer más de ese archivo devolverá una cadena vacía $''$$''$ porque se está intentando leer desde el final del archivo.

\pythonfile{codigo/14-archivos/read_3.py}
\Ainicial{codigo/14-archivos/read_3.txt_i}
\out{codigo/14-archivos/read_3.out}

Usando el método seek(), se puede mover el \doble{cursor} (puntero) al principio del archivo.

\pythonfile{codigo/14-archivos/seek_1.py}
\Ainicial{codigo/14-archivos/seek_1.txt_i}
\out{codigo/14-archivos/seek_1.out}

Alternativamente, se le puede decir a \ttt{seek()} que se mueva a otra posición.

\pythonfile{codigo/14-archivos/seek_2.py}
\Ainicial{codigo/14-archivos/seek_2.txt_i}
\out{codigo/14-archivos/seek_2.out}

El método \ttt{tell()} entrega la posición actual del puntero.

\pythonfile{codigo/14-archivos/tell.py}
\Ainicial{codigo/14-archivos/tell.txt_i}
\out{codigo/14-archivos/tell.out}

Para obtener cada línea de un archivo, se puede usar la función \ttt{readlines()} para devolver una lista donde cada elemento es una línea del archivo.

\pythonfile{codigo/14-archivos/readlines.py}
\Ainicial{codigo/14-archivos/readlines.txt_i}
\out{codigo/14-archivos/readlines.out}

Esto también se puede hacer usando un bucle for.

\pythonfile{codigo/14-archivos/for_readlines.py}
\Ainicial{codigo/14-archivos/for_readlines.txt_i}
\out{codigo/14-archivos/for_readlines.out}

\section{Escritura de archivos}

Para escribir sobre archivos se utiliza el método write(), el cual escribe una cadena en un archivo.

\pythonfile{codigo/14-archivos/write_1.py}
\Ainicial{codigo/14-archivos/write_1.txt_i}
\Afinal{codigo/14-archivos/write_1.txt_f}

Cuando un archivo es abierto en modo de escritura, el contenido existente del archivo es borrado.

\pythonfile{codigo/14-archivos/write_2.py}
\Ainicial{codigo/14-archivos/write_2.txt_i}
\Afinal{codigo/14-archivos/write_2.txt_f}

Para evitar que esto ocurra, se puede usar el modo anexo \doble{a}.

\pythonfile{codigo/14-archivos/write_3.py}
\Ainicial{codigo/14-archivos/write_3.txt_i}
\Afinal{codigo/14-archivos/write_3.txt_f}

El método \ttt{write()} devuelve el número de bytes escritos en un archivo, si su llamada es exitosa.

\pythonfile{codigo/14-archivos/write_4.py}
\Ainicial{codigo/14-archivos/write_4.txt_i}
\Afinal{codigo/14-archivos/write_4.txt_f}
\out{codigo/14-archivos/write_4.out}

Para escribir algo que no sea un string, necesita ser convertido primero a un string.

\pythonfile{codigo/14-archivos/write_5.py}
\Ainicial{codigo/14-archivos/write_5.txt_i}
\Afinal{codigo/14-archivos/write_5.txt_f}

\section{Declaración with}

Es buena práctica evitar gastar recursos asegurándose de que los archivos sean siempre cerrados después de utilizarlos. Una forma de hacer esto es utilizando \ttt{try-finally}.

\pythonfile{codigo/14-archivos/close_seguro.py}

Esto asegura que el archivo sea siempre cerrado incluso si ocurre un error.
Sin embargo, existe una forma más cómoda de hacer esto usando declaraciones with.\smallskip

Una forma alternativa de trabajar con archivos es utilizando declaraciones with. Esto crea una variable temporal (a menudo llamada f), la cual solo es accesible en el bloque indentado de la declaración with.

\pythonfile{codigo/14-archivos/with_1.py}

El archivo se cierra automáticamente al final de la declaración with, incluso si ocurren excepciones dentro de ella.\smallskip

Comparado con la forma tradicional de trabajar con archivos (open-close), usar \ttt{with} tiene el inconveniente de que los archivos tienen que volverse a abrir cada vez que se quiera trabajar con ellos.\smallskip

Así, una forma más segura de trabajar con archivos puede tener la siguiente estructura:

\pythonfile{codigo/14-archivos/with_2.py}

\clearpage