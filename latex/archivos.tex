\chapter{Archivos}

\Needspace{8\baselineskip}
\section{Abrir archivos}

Python se puede usar para leer y escribir los contenidos de archivos.
Los archivos de texto son los más fáciles de manipular.

Antes de que un archivo pueda ser editado, debe ser abierto con la función \ttt{open()}.

\pythonfile{codigo/archivos/open.py}

El argumento de la función \ttt{open()} es la ruta del archivo.
Esta ruta puede ser \link{https://www.discoduroderoer.es/rutas-relativas-y-absolutas/}{relativa} (desde el directorio de trabajo) o \link{https://www.discoduroderoer.es/rutas-relativas-y-absolutas/}{absoluta}.

Si se necesita, se puede hacer \link{https://stackoverflow.com/questions/7165749/open-file-in-a-relative-location-in-python}{un procedimiento un poco más largo} para abrir un archivo relativamente desde el directorio donde se encuentra el archivo \ttt{.py}, sin importar cual sea el directorio de trabajo o sistema operativo.

\pythonfile{codigo/archivos/open_seguro.py}

En este caso, \ttt{ruta\_relativa} es lo único que debe cambiarse para abrir un archivo distinto dentro del mismo directorio.

\Needspace{8\baselineskip}
\section{Modos de apertura}

Se puede especificar el modo utilizado para abrir un archivo al pasar un segundo argumento a la función \ttt{open()}.

Los modos de apertura son:

\begin{itemize}
  \item \ttt{\qq r \qq}: Significa modo de lectura, el cual es el modo predeterminado.
  
  \pythonfile{codigo/archivos/modo_r.py}

  \item \ttt{\qq w \qq}: Significa modo de escritura, el cual sirve para reescribir los contenidos de un archivo.
  Abrir un archivo en este modo inmediatamente borra todos sus contenidos.
  
  \pythonfile{codigo/archivos/modo_w.py}

  \item \ttt{\qq x \qq}: Significa modo de creación exclusiva, se usa para crear un fichero y escribir sobre él.
  
  \pythonfile{codigo/archivos/modo_x.py}

  Entrega un error \ttt{FileExistsError} si el archivo ya existe y no está vacío.

  \pythonfile{codigo/archivos/error_modo_x.py}
  \Ainicial{codigo/archivos/error_modo_x.txt_i}
  \out{codigo/archivos/error_modo_x.out}

  \item \ttt{\qq a \qq}: significa modo de anexo, para agregar nuevo contenido al final de un archivo.

  \pythonfile{codigo/archivos/modo_a.py}

\end{itemize}

Además de los modos de apertura, también existen los modos en los que se muestra la información:

\begin{itemize}
  \item \ttt{\qq t \qq}: significa abrir el archivo en modo texto (predeterminado).
  No es necesario escribirlo.
  
  \item \ttt{\qq b \qq}: significa abrir el archivo en modo binario, que es utilizado para archivos que no son de texto (tales como archivos de imágenes o sonido).
  Se debe combinar con alguno de los modos anteriores.

  \pythonfile{codigo/archivos/modo_b.py}

\end{itemize}

\Needspace{8\baselineskip}
\section{Extensión de modos de apertura}

Se puede añadir el signo \ttt{\qq + \qq} a cualquiera de los modos de arriba para darles acceso adicional a archivos.

La siguiente tabla muestra el funcionamiento de cada modo:
\medskip\medskip

% tabla 1

\makebox[\textwidth][c]{
  \begin{tabular}{ l l l l l l l l l }
    \toprule
    & \ttt{r} & \ttt{r+} & \ttt{w} & \ttt{w+} & \ttt{x} & \ttt{x+} & \ttt{a} & \ttt{a+} \\
    \midrule
    Lee el archivo & \si & \si & \no & \si & \no & \si & \no & \si \\
    Escribe en el archivo & \no & \si & \si & \si & \si & \si & \si & \si \\
    Crea el archivo si no existe & \no & \no & \si & \si & \si & \si & \si & \si \\
    Borra los contenidos del archivo & \no & \no & \si & \si & \noo & \noo & \no & \no \\
    Posición del cursor & \inicio & \inicio & \inicio & \inicio & \inicio & \inicio & \final & \final \\
    \bottomrule
  \end{tabular}
}
\medskip\medskip

\textcolor{Crimson}{*}: Levanta una excepción \ttt{FileExistsError} si el archivo ya tiene contenido.

\Needspace{8\baselineskip}
\section{Cierre de archivos}

Una vez que un archivo haya sido abierto y utilizado, es un buen hábito cerrarlo.
Esto se logra con el método \ttt{close()} de un objeto archivo.

\pythonfile{codigo/archivos/close.py}

A nivel de sistemas operativos, cada proceso tiene un límite en la cantidad de archivos que puede abrir simultáneamente.
Si un programa se ejecuta por mucho tiempo y podría abrir muchos archivos, se debe tener cuidado.

\Needspace{8\baselineskip}
\section{Lectura de archivos}

Los contenidos de un archivo que ha sido abierto en modo texto pueden ser leídos utilizando el método \ttt{read()}.

\pythonfile{codigo/archivos/read_1.py}
\Ainicial{codigo/archivos/read_1.txt_i}
\out{codigo/archivos/read_1.out}

Para leer sólo una determinada parte de un archivo, se puede proveer un número como argumento a la función \ttt{read()}.
Esto determina el número de bytes que deberían ser leídos.

Se pueden hacer más llamadas a \ttt{read()} en el mismo objeto archivo para leer más de él byte por byte.
Si no se le pasan argumentos, o si el argumento es negativo, \ttt{read()} devuelve el resto del archivo.

\pythonfile{codigo/archivos/read_2.py}
\Ainicial{codigo/archivos/read_2.txt_i}
\out{codigo/archivos/read_2.out}

El ejemplo de arriba primero mostrará los bytes del \ttt{1} al \ttt{16}, después del \ttt{17} al \ttt{24}, del \ttt{25} al \ttt{28}, y del \ttt{29} al último byte.

Luego de que todos los contenidos de un archivo hayan sido leídos, cualquier intento de leer más de ese archivo devolverá una cadena vacía \ttt{\qq \qq} porque se está intentando leer desde el final del archivo.

\pythonfile{codigo/archivos/read_3.py}
\Ainicial{codigo/archivos/read_3.txt_i}
\out{codigo/archivos/read_3.out}

Usando el método \ttt{seek()}, se puede mover el \doble{cursor} (puntero) al principio del archivo.

\pythonfile{codigo/archivos/seek_1.py}
\Ainicial{codigo/archivos/seek_1.txt_i}
\out{codigo/archivos/seek_1.out}

Alternativamente, se le puede decir a \ttt{seek()} que se mueva a otra posición.

\pythonfile{codigo/archivos/seek_2.py}
\Ainicial{codigo/archivos/seek_2.txt_i}
\out{codigo/archivos/seek_2.out}

El método \ttt{tell()} entrega la posición actual del puntero.

\pythonfile{codigo/archivos/tell.py}
\Ainicial{codigo/archivos/tell.txt_i}
\out{codigo/archivos/tell.out}

Para obtener cada línea de un archivo, se puede usar la función \ttt{readlines()} para devolver una lista donde cada elemento es una línea del archivo.

\pythonfile{codigo/archivos/readlines.py}
\Ainicial{codigo/archivos/readlines.txt_i}
\out{codigo/archivos/readlines.out}

Esto también se puede hacer usando un bucle \ttt{for}.

\pythonfile{codigo/archivos/for_readlines.py}
\Ainicial{codigo/archivos/for_readlines.txt_i}
\out{codigo/archivos/for_readlines.out}

\Needspace{8\baselineskip}
\section{Escritura de archivos}

Para escribir sobre archivos se utiliza el método \ttt{write()}, el cual escribe una cadena en un archivo.

\pythonfile{codigo/archivos/write_1.py}
\Ainicial{codigo/archivos/write_1.txt_i}
\Afinal{codigo/archivos/write_1.txt_f}

Cuando un archivo es abierto en modo de escritura, el contenido existente del archivo es borrado.

\pythonfile{codigo/archivos/write_2.py}
\Ainicial{codigo/archivos/write_2.txt_i}
\Afinal{codigo/archivos/write_2.txt_f}

Para evitar que esto ocurra, se puede usar el modo anexo \ttt{\qq a \qq}.

\pythonfile{codigo/archivos/write_3.py}
\Ainicial{codigo/archivos/write_3.txt_i}
\Afinal{codigo/archivos/write_3.txt_f}

El método \ttt{write()} devuelve el número de bytes escritos en un archivo, si su llamada es exitosa.

\pythonfile{codigo/archivos/write_4.py}
\Ainicial{codigo/archivos/write_4.txt_i}
\Afinal{codigo/archivos/write_4.txt_f}
\out{codigo/archivos/write_4.out}

Para escribir algo que no sea un string, necesita ser convertido primero usando el método \ttt{str()}.

\pythonfile{codigo/archivos/write_5.py}
\Ainicial{codigo/archivos/write_5.txt_i}
\Afinal{codigo/archivos/write_5.txt_f}

\Needspace{8\baselineskip}
\section{Declaración with}

Es buena práctica evitar gastar recursos asegurándose de que los archivos sean siempre cerrados después de utilizarlos.
Una forma de hacer esto es utilizando \ttt{try-finally}.

\pythonfile{codigo/archivos/close_seguro.py}

Esto asegura que el archivo sea siempre cerrado incluso si ocurre un error.
Sin embargo, existe una forma más cómoda de hacer esto usando declaraciones \ttt{with}.

Al usar declaraciones \ttt{with}, se crea una variable temporal (a menudo llamada \ttt{f}), la cual solo es accesible en el bloque indentado de la declaración \ttt{with}.

\pythonfile{codigo/archivos/with_1.py}

El archivo se cierra automáticamente al final de la declaración \ttt{with}, incluso si ocurren excepciones dentro de ella.

Así, una forma más segura de trabajar con archivos puede tener la siguiente estructura:

\pythonfile{codigo/archivos/with_2.py}

Comparado con la forma tradicional de trabajar con archivos (usar \ttt{open()} y después \ttt{close()}), usar \ttt{with} tiene el inconveniente de que los archivos tienen que volverse a abrir cada vez que se quiera trabajar con ellos.

Si se necesita mantener un archivo abierto por mucho tiempo, no se recomienda usar \ttt{with}.

\clearpage