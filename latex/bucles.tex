\chapter{Bucles}

\section{Bucles while}

Un bucle \ttt{while} se usa para repetir un bloque de código varias veces, siempre que se cumpla cierta condición. Generalmente se usan con una variable como contador.

\pythonfile{codigo/7-bucles/while_1.py}
\out{codigo/7-bucles/while_1.out}

El código en el cuerpo de un bucle \ttt{while} se ejecuta repetidamente. Esto se llama iteración.\smallskip

Se puede realizar cualquier operación con el contador dentro del bucle.

\pythonfile{codigo/7-bucles/while_2.py}
\out{codigo/7-bucles/while_2.out}

\pythonfile{codigo/7-bucles/while_3.py}
\out{codigo/7-bucles/while_3.out}

\section{Bucles infinitos}

Si la condición a evaluar es siempre verdadera, el bucle se ejecutará indefinidamente. Esto se llama bucle infinito.

\pythonfile{codigo/7-bucles/while_infinito.py}
\out{codigo/7-bucles/while_infinito.out}

No se recomienda ejecutar este código. Si por algún motivo se ejecuta, se puede interrumpir su ejecución enviando un \ttt{KeyboardInterrupt} al usar la combinación de teclas \ttt{Ctrl+C}.

\section{Declaración break}

Para finalizar un bucle \ttt{while} prematuramente, se puede usar la declaración \ttt{break}.

\pythonfile{codigo/7-bucles/break.py}
\out{codigo/7-bucles/break.out}

El uso de la declaración \ttt{break} fuera de un bucle provoca un error.

\pythonfile{codigo/7-bucles/error_break.py}
\out{codigo/7-bucles/error_break.out}

\section{Declaración continue}

Otra declaración que se puede utilizar dentro de los bucles es \ttt{continue}. A diferencia de \ttt{break}, \ttt{continue} salta de nuevo a la parte superior del bucle, en lugar de detenerlo. Básicamente, la declaración \ttt{continue} detiene la iteración actual y continúa con la siguiente.

\pythonfile{codigo/7-bucles/continue.py}
\out{codigo/7-bucles/continue.out}

Al igual que la declaración \ttt{break}, usar \ttt{continue} fuera de un bucle provoca un error.

\pythonfile{codigo/7-bucles/error_continue.py}
\out{codigo/7-bucles/error_continue.out}

\section{Bucle for con listas}

Otro tipo de bucle es el bucle \ttt{for}. Estos bucles se pueden usar para iterar sobre una lista. Las instrucciones dentro de su bloque de código se ejecutarán para cada elemento de la lista.

\pythonfile{codigo/7-bucles/for_1.py}
\out{codigo/7-bucles/for_1.out}

También se puede usar para iterar sobre cadenas.

\pythonfile{codigo/7-bucles/for_2.py}
\out{codigo/7-bucles/for_2.out}

De manera similar a los bucles \ttt{while}, las declaraciones \ttt{break} y \ttt{continue} se pueden utilizar en los bucles \ttt{for} para detener el bucle o saltar a la siguiente iteración.

\pythonfile{codigo/7-bucles/for_3.py}
\out{codigo/7-bucles/for_3.out}

\section{Rangos}

La función \ttt{range()} devuelve una secuencia de números.\smallskip

Si se llama a \ttt{range()} con un argumento, produce un objeto con valores desde \ttt{0} hasta el antecesor del número especificado.

\pythonfile{codigo/7-bucles/range_solo.py}
\out{codigo/7-bucles/range_solo.out}

Para mostrar los números dentro de un objeto range, debe convertirse a una lista usando el método \ttt{list()}.

\pythonfile{codigo/7-bucles/range_1.py}
\out{codigo/7-bucles/range_1.out}

Si se llama con 2 argumentos, produce valores desde el primero hasta el antecesor del segundo.

\pythonfile{codigo/7-bucles/range_2.py}
\out{codigo/7-bucles/range_2.out}

Si el primer número es \ttt{0}, el resultado es equivalente al de ingresar sólo 1 argumento.

\pythonfile{codigo/7-bucles/range_3.py}
\out{codigo/7-bucles/range_3.out}

El rango puede tener un tercer argumento, que determina el intervalo de la secuencia producida, también llamado como paso. Este número puede ser positivo (incremento) o negativo (decremento).

\pythonfile{codigo/7-bucles/range_4.py}
\out{codigo/7-bucles/range_4.out}

\section{Rangos imposibles}

Intentar almacenar rangos \doble{imposibles} no mostrará ningún error. El rango simplemente no tendrá ningún elemento, lo cual es más fácil de ver al convertirlo en una lista.

\pythonfile{codigo/7-bucles/rango_imposible.py}
\out{codigo/7-bucles/rango_imposible.out}

\section{Bucle for en rangos}

Una forma común de usar los bucles \ttt{for} es iterando sobre rangos. Esto permite hacer bucles de muchas formas.

\pythonfile{codigo/7-bucles/for_rango.py}
\out{codigo/7-bucles/for_rango.out}

No es necesario llamar \ttt{list()} en el objeto de rango cuando se usa en un bucle \ttt{for}, porque no se está indexando, por lo que no requiere una lista.\smallskip

En general, un bucle \ttt{for} puede recorrer cualquier iterable. Estos se verán más a fondo en su propio capítulo.

\clearpage