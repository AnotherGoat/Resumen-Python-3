\chapter{El módulo math}

\section{El módulo math}

\section{Constantes matemáticas}

\section{Infinito}

Para representar un número infinitamente pequeño o grande, se puede usar el método \ttt{float()} y entregarle el string \ttt{\qq inf \qq} como argumento.

\pythonfile{codigo/10-math/infinito.py}
\out{codigo/10-math/infinito.out}

\section{Funciones matemáticas predefinidas}

%TODO: mover sum(), investigar si existe avg() y parecidos

Python viene con algunas funciones matemáticas sencillas, que se pueden usar sin necesidad de importar el módulo \ttt{math}.

Para obtener la distancia entre un número y el \ttt{0} (su valor absulto), puede usarse la función \ttt{abs()}.

\pythonfile{codigo/10-math/abs.py}
\out{codigo/10-math/abs.out}

Para redondear un número a un determinado número de decimales, puede usarse la función \ttt{round()}.

\pythonfile{codigo/10-math/round.py}
\out{codigo/10-math/round.out}

%TODO: este capítulo

\section{Funciones trigonométricas}

\clearpage