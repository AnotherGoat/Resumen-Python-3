\chapter{El módulo math}

\Needspace{8\baselineskip}
\section{El módulo math}

El módulo \ttt{math} contiene funciones y constantes comúnmente usadas en matemáticas generales.

\pythonfile{codigo/math/math.py}

En este capítulo también se verán otras funciones matemáticas que están disponibles en Python por sí solo, sin necesidad de importar este módulo.

\Needspace{8\baselineskip}
\section{Constantes matemáticas}

El módulo \ttt{math} tiene 5 constantes matemáticas muy importantes.
Primero, se verán las 3 constantes numéricas que almacenan los valores de números irracionales importantes.

La constante \ttt{pi} representa el número \link{https://es.wikipedia.org/wiki/N\%C3\%BAmero\_\%CF\%80}{$\pi$}, el cual es la relación entre el perímetro y diámetro de una circunferencia.

$\pi = 3.14159...$

\pythonfile{codigo/math/pi.py}
\out{codigo/math/pi.out}

La constante \ttt{tau}, escrita en matemáticas como \link{https://es.wikipedia.org/wiki/Tau\_(2\%CF\%80)}{$\tau$}, es el doble de \ttt{pi}, y representa la relación entre el perímetro y el radio de una circunferencia.

$\tau = 2\pi$

\pythonfile{codigo/math/tau.py}
\out{codigo/math/tau.out}

La constante \ttt{e}  representa el \link{https://es.wikipedia.org/wiki/N\%C3\%BAmero\_e}{número de Euler}, otro de los números irracionales más importantes.

$e = 2.71828...$

\pythonfile{codigo/math/e.py}
\out{codigo/math/e.out}

\Needspace{8\baselineskip}
\section{Infinito}

Para representar un número \link{https://es.wikipedia.org/wiki/Infinito}{infinito}, se puede usar el método \ttt{float()} y entregarle el string \ttt{\qq inf \qq} como argumento.

$\infty$

\pythonfile{codigo/math/infinito.py}
\out{codigo/math/infinito.out}

Una forma alternativa de hacerlo es usando la constante \ttt{inf} del módulo \ttt{math}.

\pythonfile{codigo/math/inf.py}
\out{codigo/math/inf.out}

Ambas formas de hacerlo entregan exactamente el mismo resultado.

\pythonfile{codigo/math/igualdad_inf.py}
\out{codigo/math/igualdad_inf.out}

\Needspace{8\baselineskip}
\section{NaN}

La constante \ttt{nan} o \link{https://es.wikipedia.org/wiki/NaN}{NaN}, acrónimo proveniente del inglés Not a Number (no es un número) no proviene directamente de las matemáticas, pero tiene usos bastante útiles en programación.
Representa un número \doble{ilegal}.

\pythonfile{codigo/math/nan.py}
\out{codigo/math/nan.out}

Generalmente se usa para expresar resultados imposibles de calcular, como raíces negativas o indeterminaciones.

Esta constante se almacena como un float.

\pythonfile{codigo/math/tipo_nan.py}
\out{codigo/math/tipo_nan.out}

% TODO: ejemplo?

\Needspace{8\baselineskip}
\section{Funciones matemáticas predefinidas}

%TODO: mover sum(), investigar si existe avg() y parecidos

Python viene con algunas funciones matemáticas sencillas, que se pueden usar sin necesidad de importar el módulo \ttt{math}.
En algunos casos, el módulio \ttt{math} provee versiones más completas de estas funciones.

Para obtener la distancia entre un número y el \ttt{0} (su valor absulto), puede usarse la función \ttt{abs()}.

\pythonfile{codigo/math/abs.py}
\out{codigo/math/abs.out}

\Needspace{8\baselineskip}
\section{Funciones matemáticas}

La función \ttt{fabs()} retorna el valor absoluto de un número, al igual que \ttt{abs()}, pero siempre retorna un float.

\pythonfile{codigo/math/fabs.py}
\out{codigo/math/fabs.out}

Como ya se ha visto, la función \ttt{sqrt()}, abreviación del inglés square root, retorna la \link{https://es.wikipedia.org/wiki/Ra\%C3\%ADz\_cuadrada}{raíz cuadrada} de un número positivo o \ttt{0}.
El resultado obtenido siempre será la raíz positiva.

\pythonfile{codigo/math/sqrt.py}
\out{codigo/math/sqrt.out}

No se puede usar con números negativos.

\pythonfile{codigo/math/error_sqrt.py}
\out{codigo/math/error_sqrt.out}

La función \ttt{factorial()} retorna el \link{https://es.wikipedia.org/wiki/Factorial}{factorial} de un número.

\pythonfile{codigo/math/funcion_factorial.py}
\out{codigo/math/funcion_factorial.out}

El factorial no está definido para los números negativos.

\pythonfile{codigo/math/error_factorial.py}
\out{codigo/math/error_factorial.out}

\Needspace{8\baselineskip}
\section{Funciones de redondeo}

Para redondear un número a un determinado número de decimales, puede usarse la función \ttt{round()}.

\pythonfile{codigo/math/round.py}
\out{codigo/math/round.out}

%TODO: este capítulo

\Needspace{8\baselineskip}
\section{Grados y radianes}

Al trabajar con ángulos, es común medirlos usando grados o radianes.
El módulo \ttt{math} tiene funciones para convertir valores entre estas unidades.

El método \ttt{radians()} convierte de grados a radianes.

\pythonfile{codigo/math/radians.py}
\out{codigo/math/radians.out}

El método \ttt{degrees()} convierte de radianes a grados.

\pythonfile{codigo/math/degrees.py}
\out{codigo/math/degrees.out}

\Needspace{8\baselineskip}
\section{Funciones trigonométricas}

El módulo \ttt{math} tiene \link{https://es.wikipedia.org/wiki/Trigonometr\%C3\%ADa}{funciones trigonométricas}, similares a las de una calculadora.
Estas funciones sólo funcionan con radianes.
Para usarlas con grados, primero deben convertirse a radianes con el método \ttt{radians()}.

Funciones trigonométricas básicas:

\pythonfile{codigo/math/trig_basicas.py}
\out{codigo/math/trig_basicas.out}

Funciones trigonométricas inversas:

\pythonfile{codigo/math/trig_inversas.py}
\out{codigo/math/trig_inversas.out}

Funciones trigonométricas hiperbólicas:

\pythonfile{codigo/math/trig_hiperbolicas.py}
\out{codigo/math/trig_hiperbolicas.out}

Se debe recordar que los resultados de estas funciones se obtienen por aproximaciones y un computador no puede representar todos los números decimales con precisión perfecta, lo que significa que puede haber pequeños errores de aproximación en los resultados.

\Needspace{8\baselineskip}
\section{Números finitos e infinitos}

\Needspace{8\baselineskip}
\section{Números complejos}

\clearpage