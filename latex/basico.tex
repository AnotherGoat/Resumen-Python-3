\chapter{Conceptos básicos}

\Needspace{8\baselineskip}
\section{Cadenas de caracteres}

Las cadenas de caracteres, también conocidas en inglés como strings, se crean introduciendo el texto entre comillas simples \ttt{\q \q} o dobles \ttt{\qq \qq}.

\pythonfile{codigo/basico/sintaxis_string.py}

El método \ttt{print()} mostrado en la introuducción recibe una cadena como parámetro y la imprime en el terminal.

\pythonfile{codigo/basico/string.py}
\out{codigo/basico/string.out}

El uso de comillas simples o dobles no afecta en ninguna forma el comportamiento del string, es decir, ambos producen el mismo resultado.

\Needspace{8\baselineskip}
\section{Comillas mixtas}

Una cadena debe empezar y terminar con comillas del mismo tipo, no se permiten comillas mixtas.

\pythonfile{codigo/basico/string_no_valido_1.py}
\out{codigo/basico/string_no_valido_1.out}
\pythonfile{codigo/basico/string_no_valido_2.py}
\out{codigo/basico/string_no_valido_2.out}

Los errores que ocurren aquí se conocen como excepciones.

\Needspace{8\baselineskip}
\section{Comentarios}

Los comentarios son anotaciones en el código utilizadas para hacerlo más fácil de entender.
No afectan la ejecución del código.

En Python, los comentarios comienzan con el símbolo \ttt{\#}.
Todo el texto luego de este \ttt{\#} (dentro de la misma línea) es ignorado.

\sintaxis{codigo/basico/sintaxis_comentario.py}
\codigo{codigo/basico/comentario_1.py}
\out{codigo/basico/comentario_1.out}

La mayoría de editores de código marcan los comentarios con su propio color, distinto al del resto de código.

\codigo{codigo/basico/comentario_2.py}
\out{codigo/basico/comentario_2.out}

Los comentarios no tienen fines meramente explicativos.
También se pueden usar para esconder líneas de código, con el propósito de hacer más fácil revisar que funciona bien.

\codigo{codigo/basico/probando_codigo.py}
\out{codigo/basico/probando_codigo.out}

Se debe recordar borrar los comentarios de este tipo después de terminar las pruebas de código, ya que no son de utilidad y hay herramientas mejores, como \link{https://git-scm.com}{Git} (un sistema de control de versiones), para mantener guardado código antiguo en caso de que algún error ocurra.

\Needspace{8\baselineskip}
\section{Comentarios multilínea}

Python no soporta comentarios multilínea para fines generales como los tienen otros lenguajes de programación tales como C.
Para comentar varias líneas de código se debe antener un comentario a cada línea.

\codigo{codigo/basico/comentario_multilinea.py}

\Needspace{8\baselineskip}
\section{Comentarios innecesarios}

Se debe recordar no abusar del uso de comentarios.
No es conveniente llenar el código de comentarios o comentar cosas que son demasiado obvias.

\codigo{codigo/basico/comentario_innecesario.py}

La excepción a esta regla podría ser cuando se usan con fines educativos.
Por ejemplo, para introducir a alguien sin conocimiento sobre programación a Python.
Sin embargo, es importante recordar que en la práctica los comentarios innecesarios deben evitarse.

\Needspace{8\baselineskip}
\section{Números enteros}

Los números enteros (en inglés, integer) son el tipo de dato más básico en muchos lenguajes de programación, y Python no es la excepción.

\sintaxis{codigo/basico/sintaxis_integer.py}

Para mostrar un número en pantalla, sólo basta con usar la función \ttt{print()} y entregarle el número.
El número debe escribirse sin usar comillas y Python inferirá su tipo automáticamente.

\codigo{codigo/basico/numero_1.py}
\out{codigo/basico/numero_1.out}

La función \ttt{print()} puede recibir tanto cadenas como números, y mucho tipos de datos más, que se verán más tarde

\Needspace{8\baselineskip}
\section{Tamaño máximo de números}

A diferencia de otros lenguajes de programación, los números en Python 3 son de largo arbitrario, en otras palabras, soportan \link{https://stackoverflow.com/questions/13795758/what-is-sys-maxint-in-python-3}{cualquier cantidad de dígitos}.
Esto hace que los cálculos con números grandes sean mucho más convenientes, pero reduce el rendimiento.

\codigo{codigo/basico/numero_2.py}
\out{codigo/basico/numero_2.out}

En las versiones anteriores a Python 3, si tenían un largo máximo.

\Needspace{8\baselineskip}
\section{Floats}
    
Para representar números racionales o que no son enteros, se usa el tipo de dato float o punto flotante.
Se pueden crear directamente ingresando un número con un punto decimal.

\sintaxis{codigo/basico/sintaxis_float.py}
\codigo{codigo/basico/float.py}
\out{codigo/basico/float.out}

\Needspace{8\baselineskip}
\section{Fracciones}

Una forma alternativa de ingresar números decimales es escribirlos como fracciones, utilizando el operador \ttt{/}.

\sintaxis{codigo/basico/sintaxis_fraccion.py}
\codigo{codigo/basico/float_fraccion.py}
\out{codigo/basico/float_fraccion.out}

En realidad, lo que ocurre aquí es una consecuencia de la operación de división, la cual es equivalente a una fracción.

\Needspace{8\baselineskip}
\section{Variables}

Una variable permite almacenar un valor asignándole un nombre, el cual puede ser usado para referirse al valor más adelante en el programa.
Para asignar una variable, se usa el signo de igualdad \ttt{=}.

\pythonfile{codigo/basico/sintaxis_variable.py}
\pythonfile{codigo/basico/variable.py}
\out{codigo/basico/variable.py}

\clearpage